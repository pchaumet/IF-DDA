\chapter{Quleques exemples}\label{chaptest}
\markboth{\uppercase{Fichiers de test}}{\uppercase{Fichiers de test}}

\minitoc

\section{Introduction}

Dans bin/tests est dispos� un fichier options.db3. Si on le copie un
directory en dessous ``cp options.db3 ../.'', quand on lance le code
apr�s un load il appara�t quatre configurations tests qui permettent de
voir toutes les options en action.

\section{Test1}

Le but du test1 est de tester un cas simple et de nombreuses options
du code afin de les valider.  La Fig.~\ref{test1conf} montre les
options de la configuration choisie.


%%%%%%%%%%%%%%%%%%%%%%%%%%%%%%%%%%%%%%%%%%%%%%%
\begin{figure}[H]
\begin{center}
  \includegraphics*[width=15.0cm,draft=false]{test1conf.eps}
\end{center}
\caption{Test1: configuration choisie.}
\label{test1conf}
\end{figure}
%%%%%%%%%%%%%%%%%%%%%%%%%%%%%%%%%%%%%%%%%%%%%%%

Les Figures suivantes montrent les r�sultats obtenus.  Les trac�s sont
effectu�s avec Matlab et ce sont directement les fichiers eps tir�s du
script ifdda.m qui sont utilis�s, mais ceux-ci peuvent bien s�r �tre 
r�alis�s avec l'interface graphique int�gr�e. L'avantage de matlab dans
ce cas est de donner toutes les figures d'un seul coup.
%%%%%%%%%%%%%%%%%%%%%%%%%%%%%%%%%%%%%%%%%%%%%%%
\begin{figure}[H]
\begin{center}
  \includegraphics*[width=15.0cm,draft=false]{test1local.eps}
\end{center}
\caption{Module du champ local dans le plan $(x,y)$.}
%\label{test1res1}
\end{figure}
%%%%%%%%%%%%%%%%%%%%%%%%%%%%%%%%%%%%%%%%%%%%%%%
%%%%%%%%%%%%%%%%%%%%%%%%%%%%%%%%%%%%%%%%%%%%%%%
\begin{figure}[H]
\begin{center}
  \includegraphics*[width=15.0cm,draft=false]{test1macro.eps}
\end{center}
\caption{Module du champ macroscopique dans le plan $(x,y)$.}
%\label{test1res1}
\end{figure}
%%%%%%%%%%%%%%%%%%%%%%%%%%%%%%%%%%%%%%%%%%%%%%%
Le champ incident �tant polaris� suivant la composante $y$ (TE), il
est clair que la composante $y$ du champ � l'int�rieur de la sph�re
est la plus forte.

%%%%%%%%%%%%%%%%%%%%%%%%%%%%%%%%%%%%%%%%%%%%%%%
\begin{figure}[H]
\begin{center}
  \includegraphics*[width=15.0cm,draft=false]{test1poynting2d.eps}
\end{center}
\caption{Champ rayonn� par l'objet.}
%\label{test1res1}
\end{figure}
%%%%%%%%%%%%%%%%%%%%%%%%%%%%%%%%%%%%%%%%%%%%%%%
%%%%%%%%%%%%%%%%%%%%%%%%%%%%%%%%%%%%%%%%%%%%%%%
\begin{figure}[H]
\begin{center}
\begin{tabular}{cc}
  \includegraphics*[width=7.0cm,draft=false]{test1force2d.eps}
&  \includegraphics*[width=9.0cm,draft=false]{test1force3d.eps}
\end{tabular}

\end{center}
\caption{Force optique dans la plan $(x,y)$ et en trois D.}
%\label{test1res2}
\end{figure}
%%%%%%%%%%%%%%%%%%%%%%%%%%%%%%%%%%%%%%%%%%%%%%%
%%%%%%%%%%%%%%%%%%%%%%%%%%%%%%%%%%%%%%%%%%%%%%%
\begin{figure}[H]
\begin{center}
\begin{tabular}{cc}
  \includegraphics*[width=7.0cm,draft=false]{test1torque2d.eps}
&  \includegraphics*[width=9.0cm,draft=false]{test1torque3d.eps}
\end{tabular}

\end{center}
\caption{Couple optique dans la plan $(x,y)$ et en trois D.}
%\label{test1res2}
\end{figure}
%%%%%%%%%%%%%%%%%%%%%%%%%%%%%%%%%%%%%%%%%%%%%%%

%%%%%%%%%%%%%%%%%%%%%%%%%%%%%%%%%%%%%%%%%%%%%%%
\begin{figure}[H]
\begin{center}
\begin{tabular}{ccc}
  \includegraphics*[width=5.0cm,draft=false]{test1fourier.eps}
& \includegraphics*[width=5.0cm,draft=false]{test1image.eps}
&  \includegraphics*[width=5.0cm,draft=false]{test1imageinc.eps}

\end{tabular}

\end{center}
\caption{Microscopie en tramsmission: Module du champ diffract� dans
  le domaine de Fourier (gauche), module du champ diffract� dans le
  plan image (milieu), et module du champ total dans le plan image
  (droite).}
\end{figure}
%%%%%%%%%%%%%%%%%%%%%%%%%%%%%%%%%%%%%%%%%%%%%%%

\section{Test2}

Le but du test2 est de tester un cas simple et de nombreuses options
du code afin de les valider.  La Fig.~\ref{test2conf} montre les
options de la configuration choisie. L'�clairement est fait par deux
ondes planes qui interf�rent.


%%%%%%%%%%%%%%%%%%%%%%%%%%%%%%%%%%%%%%%%%%%%%%%
\begin{figure}[H]
\begin{center}
  \includegraphics*[width=15.0cm,draft=false]{test2conf.eps}
\end{center}
\caption{Test2: configuration choisie.}
\label{test2conf}
\end{figure}
%%%%%%%%%%%%%%%%%%%%%%%%%%%%%%%%%%%%%%%%%%%%%%%


%%%%%%%%%%%%%%%%%%%%%%%%%%%%%%%%%%%%%%%%%%%%%%%
\begin{figure}[H]
\begin{center}
\begin{tabular}{ccc}
  \includegraphics*[width=7.0cm,draft=false]{test2dipolepos.eps}
& \includegraphics*[width=7.0cm,draft=false]{test2epsilon.eps}
\end{tabular}

\end{center}
\caption{Repr�sentation tridimensionnelle de l'objet (gauche) et carte
  de permittivit� dans le plan $(x,y)$ (droite).}
\end{figure}
%%%%%%%%%%%%%%%%%%%%%%%%%%%%%%%%%%%%%%%%%%%%%%%


Les Figures suivantes montrent les r�sultats obtenus.
%%%%%%%%%%%%%%%%%%%%%%%%%%%%%%%%%%%%%%%%%%%%%%%
\begin{figure}[H]
\begin{center}
  \includegraphics*[width=15.0cm,draft=false]{test2incident.eps}
\end{center}
\caption{Module du champ incident dans le plan $(x,y)$.}
%\label{test1res1}
\end{figure}
%%%%%%%%%%%%%%%%%%%%%%%%%%%%%%%%%%%%%%%%%%%%%%%
%%%%%%%%%%%%%%%%%%%%%%%%%%%%%%%%%%%%%%%%%%%%%%%
\begin{figure}[H]
\begin{center}
  \includegraphics*[width=15.0cm,draft=false]{test2local.eps}
\end{center}
\caption{Module du champ local dans le plan $(x,y)$.}
%\label{test1res1}
\end{figure}
%%%%%%%%%%%%%%%%%%%%%%%%%%%%%%%%%%%%%%%%%%%%%%%
%%%%%%%%%%%%%%%%%%%%%%%%%%%%%%%%%%%%%%%%%%%%%%%
\begin{figure}[H]
\begin{center}
  \includegraphics*[width=15.0cm,draft=false]{test2macro.eps}
\end{center}
\caption{Module du champ macroscopique dans le plan $(x,y)$.}
%\label{test1res1}
\end{figure}
%%%%%%%%%%%%%%%%%%%%%%%%%%%%%%%%%%%%%%%%%%%%%%%

%%%%%%%%%%%%%%%%%%%%%%%%%%%%%%%%%%%%%%%%%%%%%%%
\begin{figure}[H]
\begin{center}
  \includegraphics*[width=15.0cm,draft=false]{test2poynting2d.eps}
\end{center}
\caption{Champ rayonn� par l'objet.}
%\label{test1res1}
\end{figure}
%%%%%%%%%%%%%%%%%%%%%%%%%%%%%%%%%%%%%%%%%%%%%%%


%%%%%%%%%%%%%%%%%%%%%%%%%%%%%%%%%%%%%%%%%%%%%%%
\begin{figure}[H]
\begin{center}
\begin{tabular}{ccc}
  \includegraphics*[width=7.0cm,draft=false]{test2fourier.eps}
& \includegraphics*[width=7.0cm,draft=false]{test2image.eps}
\end{tabular}

\end{center}
\caption{Microscopie en r�flexion: Module du champ diffract� dans le
  domaine de Fourier (gauche), module du champ diffract� dans le plan
  image (droite).}
\end{figure}
%%%%%%%%%%%%%%%%%%%%%%%%%%%%%%%%%%%%%%%%%%%%%%%




\section{Test3}

Le but du test3 est de tester la microscopie en champ sombre et champ
brillant en transmission. On �tudie une sph�re de 500~nm de rayon et
de permittivit� 1.5.


%%%%%%%%%%%%%%%%%%%%%%%%%%%%%%%%%%%%%%%%%%%%%%%
\begin{figure}[H]
\begin{center}
  \includegraphics*[width=15.0cm,draft=false]{test3conf.eps}
\end{center}
\caption{Test3: configuration choisie.}
\label{test3conf}
\end{figure}
%%%%%%%%%%%%%%%%%%%%%%%%%%%%%%%%%%%%%%%%%%%%%%%


%%%%%%%%%%%%%%%%%%%%%%%%%%%%%%%%%%%%%%%%%%%%%%%
\begin{figure}[H]
\begin{center}
\begin{tabular}{ccc}
 \includegraphics*[width=5.0cm,draft=false]{test3angleincbf.eps}
&  \includegraphics*[width=5.0cm,draft=false]{test3imagewf.eps}
& \includegraphics*[width=5.0cm,draft=false]{test3imageincwf.eps}
\end{tabular}

\end{center}
\caption{Microscopie en transmission: incident pris pour cr�er l'image
  (gauche). Module du champ diffract� dans le plan image (milieu),
  module du champ total dans le plan image (droite).}
\end{figure}
%%%%%%%%%%%%%%%%%%%%%%%%%%%%%%%%%%%%%%%%%%%%%%%


\section{Test4}

Le but du test4 est de tester la microscopie en champ sombre et champ
brillant entransmission. On �tudie une sph�re de 500~nm de rayon est
de permittivt� 1.5.


%%%%%%%%%%%%%%%%%%%%%%%%%%%%%%%%%%%%%%%%%%%%%%%
\begin{figure}[H]
\begin{center}
  \includegraphics*[width=15.0cm,draft=false]{test4conf.eps}
\end{center}
\caption{Test4: configuration choisie.}
\label{test4conf}
\end{figure}
%%%%%%%%%%%%%%%%%%%%%%%%%%%%%%%%%%%%%%%%%%%%%%%


%%%%%%%%%%%%%%%%%%%%%%%%%%%%%%%%%%%%%%%%%%%%%%%
\begin{figure}[H]
\begin{center}
\begin{tabular}{cc}
 \includegraphics*[width=7.0cm,draft=false]{test4angleincdf.eps}
&  \includegraphics*[width=7.0cm,draft=false]{test4imagewf.eps}
\end{tabular}

\end{center}
\caption{Microscopie en r�flexion: incident pris pour cr�er l'image
  (gauche). Module du champ diffract� dans le plan image (droite).}
\end{figure}
%%%%%%%%%%%%%%%%%%%%%%%%%%%%%%%%%%%%%%%%%%%%%%%
