\chapter{Generality}\label{chap1}
\markboth{\uppercase{Generality}}{\uppercase{Generality}}

\minitoc

\section{Introduction}


This software computes the diffraction of an electromagnetic wave by a
three-dimensional object. This interaction is taken into account
rigorously by solving the Maxwell's equations, but can also do with
the approximation of Born, Rytov or the BPM. The code has an
user-friendly interface and allows you to choose canonical objects
(sphere, cube, ...) as well as predefined incident waves (plane wave,
Gaussian beam, ...) or aribitrary objects and incidents waves. After
by drop-down menus, it is easy to study cross sections, optical forces
and torques, diffraction near field and far field as well as
microsocopy.


There are numerous methods that enable the study of the diffraction of
an electromagnetic wave by an object of arbitrary form and relative
permittivity. We are not going here to set up an exhaustive list of
these methods, but the curious reader may refer to the article by
F. M. Kahnert who details the advantages and weaknesses of the most
common methods.~\cite{Kahnert_JQSRT_03} 

The method we use is called coupled dipoles method (CDM) or the
discrete dipole approximation (DDA). This method is a volume method,
because the diffracted field is obtained from an integral, the support
of which is the volume of the considered object.  It had been
introduced by E. M. Purcell and C. R. Pennypacker in 1973, in order to
study the scattering of light by grains in interstellar
medium.~\cite{Purcell_AJ_73} The DDA has been subsequently widened to
objects in presence of a plane substrate or in a multilayer system,
see for instance Ref.~[\onlinecite{Rahmani_PRA_97}]. These past few
years, we endeavoured, on the one hand, to widen the DDA to more
complex geometries (grating with or without any default) and, on the
other hand, to increase its precision.  These improvements tend to
make this chapter a little technical, but they are going to be applied
in the next chapters. Before studying more in details the last
improvements of DDA, though, let us remind first of its principle.

\section{The principle of discrete dipole approximation}

Take an object of arbitrary form and relative permittivity in a
homogeneous space that we suppose here being the vacuum. This object
is submitted to an incident electromagnetic wave of wavelength
$\lambda$ ($k_0=2\pi/\lambda$).  The principle of the DDA consists in
representing the object as a set of $N$ small cubes of an edge $a$ [by
little, we mean smaller than the wavelength in the object :
$a\ll \lambda/\sqrt{\varepsilon}$ (Fig.~\ref{discretisation})].
%%%%%%%%%%%%%%%%%%%%%%%%%%%%%%%%%%%%%%%%%%%%%%%%%%%%%%%%%%%%%%%%%%%%%%
\begin{figure}
\begin{center}
\includegraphics*[draft=false,width=150mm]{discretisation.eps}
\caption{Principle of the DDA : the object under study (on the left)
  is discretized in a set of small dipoles (on the right)l.}
\label{discretisation}
\end{center}
\end{figure}
%%%%%%%%%%%%%%%%%%%%%%%%%%%%%%%%%%%%%%%%%%%%%%%%%%%%%%%%%%%%%%%%%%%%%%
Each one of the small cubes under the action of the incident wave is
going to get polarized, and as such, to acquire a dipolar moment,
whose value is going to depend on the incident field and on its
interaction with its neighbours. The local field of a dipole located
at $\ve{r}_i$, $\ve{E}(\ve{r}_i)$, is the sum of the incident wave and
the field radiated by the other $N-1$ dipoles :
%%%%%%%%%%%%%%%%%%%%%%%%%%%%%%%%%%%%%%%%%%%%%%%%%%%%%%%%%%%%%%%%%%%%%%
\be \label{cdms} \ve{E}(\ve{r}_i)=\ve{E}_0(\ve{r}_i)+\sum_{j=1,i\neq
j}^{N} \ve{T}(\ve{r}_i,\ve{r}_j)\alpha(\ve{r}_j)\ve{E}(\ve{r}_j). \ee
%%%%%%%%%%%%%%%%%%%%%%%%%%%%%%%%%%%%%%%%%%%%%%%%%%%%%%%%%%%%%%%%%%%%%%
$\ve{E}_0$ is the incident wave, $\ve{T}$ the linear susceptibility of
the field in homogeneous space:
%%%%%%%%%%%%%%%%%%%%%%%%%%%%%%%%%%%%%%%%%%%%%%%%%%%%%%%%%%%%%%%%%%%%%%
\be \ve{T}(\ve{r}_i,\ve{r}_j)=e^{ik_0 r}
\left[\left(3\frac{\ve{r}\bigotimes\ve{r}}{r^2}- \ve{I}\right)
  \left(\frac{1}{r^3}-\frac{ik_0}{r^2}\right) +
  \left(\ve{I}-\frac{\ve{r}\bigotimes\ve{r}}{r^2}\right)
  \frac{k_0^2}{r}\right] \ee
%%%%%%%%%%%%%%%%%%%%%%%%%%%%%%%%%%%%%%%%%%%%%%%%%%%%%%%%%%%%%%%%%%%%%%
with $\ve{I}$ the unity matrix and
$\ve{r}=\ve{r}_i-\ve{r}_j$. $\alpha$ is the polarizability of each
discretization element obtained from the Clausius-Mossotti
relation. Note that the polarizability $\alpha$, in order to respect
the optical theorem, needs to contain a term called the radiative
reaction term.~\cite{Draine_AJ_88} Equation~(\ref{cdms}) is valid for
$i=1,\cdots,N$, and so represents a system of $3N$ linear equations
where the local fields, $\ve{E}(\ve{r}_i)$, being the unknowns. Once
the system of linear equation is solved, the field scattered by the
object at an arbitrary position $\ve{r}$ is obtained by making the sum
of all the radiated fields by each one of the dipoles :
%%%%%%%%%%%%%%%%%%%%%%%%%%%%%%%%%%%%%%%%%%%%%%%%%%%%%%%%%%%%%%%%%%%%%%
\be \label{cdmd} \ve{E}(\ve{r})=\sum_{j=1}^{N} \ve{T}(\ve{r},\ve{r}_j)
\alpha(\ve{r}_j) \ve{E}(\ve{r}_j). \ee
%%%%%%%%%%%%%%%%%%%%%%%%%%%%%%%%%%%%%%%%%%%%%%%%%%%%%%%%%%%%%%%%%%%%%%
When the object is in presence of a plane substrate or within a
multilayer system, it is just necessary to replace $\ve{T}$, by the
linear susceptibility of the referential system.

We have just presented the DDA as E. M. Purcell and C. R. Pennypacker
had presented it earlier.~\cite{Purcell_AJ_73} Note that another
method very close to the DDA does exist. This method called the method
of the moments starts from the integral equation of Lippman Schwinger,
which is strictly identical to the DDA. The demonstration of the
equivalence between these two methods being a little technical, it is
explained in Ref.~\cite{Chaumet_PRE_04}.

The advantages of the DDA are that it is applicable to objects of
arbitrary forms, inhomogeneous (that is hardly achievable in case of
surface method), and anisotropic (the polarizability associated to the
mesh becomes a tensor). The outgoing wave condition is automatically
satisfied through the linear susceptibility of the field. Finally,
note that only the object is discretized unlike the methods of finite
differences and finite elements.~\cite{Kahnert_JQSRT_03} The main
inconvenience of the DDA consists in the fast increase of computation
time together with the increase of the number of discretization
elements, {\it i.e.}, the increase in size of the system of linear
equations to be solved.  There are ways to accelerate the resolution
of a system of linear equations very important in size as the method
of conjugated gradients, but, besides all, values of $N>10^6$ in
homogeneous space are difficult to deal with.

\section{A word about the code}

The code is thought to have a user-friendly interface so that everyone
can use it without any problems including non specialists. This allows
undergraduate students to study, for example, the basics of microscopy
(Rayleigh's criteria, notion of numerical aperture, ...)  or
diffraction without any problem; and researchers, typically
biologists, having no notion of Maxwell's equations to simulate what
gives a microscope (brightfield, phase microscope, dark field, ...) in
function of the usual parameters and the object. Nevertheless, this
code can also serve physicists specializing in electromagnetism in
performing, for example, calculations of optical forces, diffraction,
cross sections, near field and this with many incident beams.

The code thus has by default a simple interface where all numeroical
parameters are hidden and where many options are then chosen by
default. But it's easy to access all Code options by checking the
Advanced Interface option. This userguide explains how to use the
advanced interface in starting with the different approaches used by
the code to solve the Maxwell equations.

Note that the usability of the code is made to the detriment of the
optimization of the RAM and the code can used large memory for large
objects.


\section{How to compile the code}

The application is based on Qt and gfortran To install it you need :
qt, qt-devel, gcc-c++ et gfortran.  Notice that there is three
versions of the code, the first one is sequential and uses FFTE (Fast
Fourier Transform in the east), the second one uses FFTW (Fast Fourier
Transform in the west) which needs openmp 4.5 minimum and the third
uses HDF5 format to save data file. Currently according to the age of
the linux you use, you have Qt4 or Qt5. The code has been tested under
the two environments, but to compile you need adapt qt4 in qt5 on
recent versions, I will note for make compact qt4(5). Then to compile:

\begin{tabular}{|c|c|c|}
  \hline
  Code par défaut & Code avec FFTW & Code avec FFTW et HDF5 \\
  \hline
  qmake-qt4 & qmake-qt4 ``CONFIG+=fftw'' & qmake-qt4 ``CONFIG+=fftw hdf5'' \\
  make & make & make \\
make install & make install & make install \\
  \hline
\end{tabular}

To run the application, cd bin, and ./cdm.


On linux system with the library FFTW, it requires to install FFTW
packages with `` dnf install * fftw * ''. For the version that uses
HDF5 file you should install the following packages ``dnf install hdf
hdf5 hdf5-static hdf5-devel''.

The code works on windows system but it is tricky to compile it if you
want to use FFTW.



\section{A word about the authors}

\begin{itemize}
\item P. C. Chaumet is Professor at Fresnel Institute of Aix-Marseille
  University, and deals with the development of the fortran source
  code.
\item A. Sentenac is research director at the CNRS, and works at
  Fresnel Institute of Aix-Marseille University, and participates to
  the development of the code connected to the far field diffraction.
\item D. Sentenac develops the convivial interface of the code.
\item G. Henry at Fresnel Institute of Aix-Marseille University works
  on the makefile and compilation on different linux system (Ubuntu,
  Fedora, etc).
\end{itemize}

\section{Licence}


Attribution-NonCommercial-ShareAlike 4.0 International (CC BY-NC-SA 4.0)

You are free to:

\begin{itemize}
\item Share - copy and redistribute the material in any medium or
  format
\item Adapt - remix, transform, and build upon the material
\end{itemize}

The licensor cannot revoke these freedoms as long as you follow the
license terms.
\begin{itemize}
\item Attribution - You must give appropriate credit, provide a link
  to the license, and indicate if changes were made. You may do so in
  any reasonable manner, but not in any way that suggests the licensor
  endorses you or your use.
\item NonCommercial - You may not use the material for commercial
  purposes.
\item ShareAlike - If you remix, transform, or build upon the
  material, you must distribute your contributions under the same
  license as the original.
\end{itemize}


\section{How to quote the code}

\begin{itemize}

\item If only the basic functions of the code are used:

P. C. {\textsc{Chaumet}}, A. {\textsc{Sentenac}}, and
A. {\textsc{Rahmani}}, \\{\it Coupled dipole method for scatterers
  with large permittivity.}\\
Phys. Rev. E {\bf 70}, 036606 (2004).

\item If the calculation of the optical forces is used, then:

P.C. {\textsc{Chaumet}}, A. {\textsc{Rahmani}},
A. {\textsc{Sentenac}}, and G. W. {\textsc{Bryant}},\\ {\it Efficient
  computation of optical forces with the coupled dipole method.}\\
Phys. Rev. E {\bf 72}, 046708 (2005).

\item If the calculation of optical couples is used:

P. C. {\textsc{Chaumet}} and C. {\textsc{Billaudeau}},\\ {\it Coupled
  dipole method to compute optical torque: Application to a
  micropropeller.}\\
J. Appl. Phys. {\bf 101}, 023106 (2007).

\item If the rigorous Gaussian beam is used:

P. C. {\textsc{Chaumet}},\\ {\it Fully vectorial highly non
  paraxial beam close to the waist.}\\
J. Opt. Soc. Am. A {\bf 23}, 3197 (2006).

\end{itemize}
