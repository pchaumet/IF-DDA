%%%%% Main pour appeler les sat %%%%%
\documentclass[french, a4paper, 11pt, oneside]{book}

%%%%%%%%%%%%%%%%%%%%%%%%%%%%%%%%%%%%%%%%%%%%%%%%%%%%%%
%%%%%% regle la mise en page, et les chapitres %%%%%%%
%%%%%%%%%%%%%%%%%%%%%%%%%%%%%%%%%%%%%%%%%%%%%%%%%%%%%%
\usepackage{fancyhdr}
\usepackage[T1]{fontenc}
\usepackage[latin1]{inputenc}
\usepackage{ae}
\usepackage[frenchb]{babel}
\usepackage[french]{minitoc}
\usepackage[french]{varioref}
\usepackage{amssymb,amsbsy,amsfonts,amsmath,subeqnarray,eqnarray}
\usepackage{amsfonts}
\usepackage{vmargin}	% gere les marges
\usepackage{color}  % gere toute les couleurs
\usepackage{picins}
\usepackage{graphicx}
\usepackage[normalem]{ulem}
\usepackage{float}
\usepackage{chngpage} 
\usepackage{wrapfig}
\usepackage{program}
\usepackage{lettrine}
\usepackage{type1cm}
\usepackage{aeguill}

\usepackage{eclbkbox}
\usepackage{atxy}
\usepackage{array}
\usepackage[titletoc]{appendix}
\usepackage{makeidx}
\usepackage{subfigure}
\usepackage{braket}
\usepackage{hyperref}


\hypersetup{
     backref=true,    %permet d'ajouter des liens dans...
     pagebackref=true,%...les bibliographies
     hyperindex=true, %ajoute des liens dans les index.
     colorlinks=true, %colorise les liens
     breaklinks=true, %permet le retour a la ligne dans les liens trop longs
     urlcolor= blue,  %couleur des hyperliens
     linkcolor= blue, %couleur des liens internes
     bookmarks=true,  %cree des signets pour Acrobat
     bookmarksopen=true,            %si les signets Acrobat sont crees,
                                    %les afficher completement.
     pdftitle={Userguide IFDDA}, %informations apparaissant dans
     pdfauthor={Patrick C. Chaumet},     %dans les informations du document
     pdfsubject={IFDDA}          %sous Acrobat.
}



%%%%%%%%% declaration pour references comme ds revtex 4 %%%%%%%
\usepackage[numbers,super,sort&compress]{natbib}
\makeatletter \DeclareRobustCommand\onlinecite{\@onlinecite}
\def\@onlinecite#1{\begingroup\let\@cite
\NAT@citenum\citealp{#1}\endgroup} \makeatother
%%%%%%%%% change numerotation des footnotes %%%%%%%%%%%%%%%%%
\renewcommand{\thefootnote}{\roman{footnote}}

% A4wide pour elargir la page....

%%%% redefini les captions
\usepackage[small]{caption2}
\renewcommand{\captionfont}{\it \small}
\renewcommand{\captionlabelfont}{\it \bf \small}
\renewcommand{\captionlabeldelim}{ :}
\setlength{\captionmargin}{20 pt}%\captionmargin
%%%%%%%%%%%%%%%%%%%%%%%%%%%%%%%%%%%%%%%%%%%%%%%%%%%%%%
%%%%%%%%%%%%%%%%% regle des marges %%%%%%%%%%%%%%%%%%%
%%%%%%%%%%%%%%%%%%%%%%%%%%%%%%%%%%%%%%%%%%%%%%%%%%%%%%
\setmargins{25mm}{16mm}{150mm}{240mm}{10mm}{5mm}{10mm}{10mm}
%           left  top   width  height head  hsep foot  fskip
\setcounter{secnumdepth}{7}
\setcounter{tocdepth}{7}
\setcounter{minitocdepth}{2}
%\setcounter{lofdepth}{2}
%\setlength{\doublerulesep}{\arrayrulewidth} %% pour les tableaux(E)
%%%%%%%%%%%%%%%%%%%%%%%%%%%%%%%%%%%%%%%%%%%%%%%%%%%%%%%

%%%%%%%%%%%%%%%%%%%%%%%%%%%%%%%%%%%%%%%%%%%%%%%%%%%%%%
%%%%%%%%%%%%%%%%%% style de la page %%%%%%%%%%%%%%%%%%
%%%%%%%%%%%%%%%%%%%%%%%%%%%%%%%%%%%%%%%%%%%%%%%%%%%%%%
\definecolor{gris}{gray}{0.50}
\pagestyle{fancy}
\fancyhf{}
\renewcommand{\chaptermark}[1]{\markboth{#1}{}}
\renewcommand{\sectionmark}[1]{\markright{\thesection\ #1}}
\fancyhead[LE,RO]{\bfseries\thepage}
\fancyhead[LO,RE]{\bfseries\footnotesize\textcolor{gris}{\rightmark}}
%%%%%%%%%%%%%%%%%%%%%%%%%%%%%%%%%%%%%%%%%%%%%%%%%%%%

%%%%%%%%%%%%%%%%%%%%%%%%%%%%%%%%%%%%%%%%%%%%%%%%%%%%
%%%%%%%%%%%%%%% style des chapitres %%%%%%%%%%%%%%%%
%%%%%%%%%%%%%%%%%%%%%%%%%%%%%%%%%%%%%%%%%%%%%%%%%%%%
\makeatletter
\def\thickhrulefill{\leavevmode \leaders \hrule height 1ex \hfill \kern \z@}
\def\@makechapterhead#1{%
  %\vspace*{50\p@}%
  \vspace*{10\p@}%
  {\parindent \z@ \centering \reset@font
        \thickhrulefill\quad
        \scshape \@chapapp{} \thechapter
        \quad \thickhrulefill
        \par\nobreak
        \vspace*{10\p@}%
        \interlinepenalty\@M
        \hrule
        \vspace*{10\p@}%
        \Huge \bfseries #1\par\nobreak
        \par
        \vspace*{10\p@}%
        \hrule
    %\vskip 40\p@
    \vskip 100\p@
  }}
\def\@makeschapterhead#1{%
  %\vspace*{50\p@}%
  \vspace*{10\p@}%
  {\parindent \z@ \centering \reset@font
        \thickhrulefill
        \par\nobreak
        \vspace*{10\p@}%
        \interlinepenalty\@M
        \hrule
        \vspace*{10\p@}%
        \Huge \bfseries #1\par\nobreak
        \par
        \vspace*{10\p@}%
        \hrule
    %\vskip 40\p@
    \vskip 100\p@
  }}
%%%%%%%%%%%%%%%%%%%%%%%%%%%%%%%%%%%%%%%%%%%%%%%%%%%%
% L'environnement changemargin d�crit ci-dessous permet de
% modifier localement les marges d'un document. Il prend deux
% arguments, la marge gauche et la marge droite (ces arguments
% peuvent prendre des valeurs n�gatives).
%%%% debut macro %%%%
\newenvironment{changemargin}[2]{\begin{list}{}{%
\setlength{\topsep}{0pt}%
\setlength{\leftmargin}{0pt}%
\setlength{\rightmargin}{0pt}%
\setlength{\listparindent}{\parindent}%
\setlength{\itemindent}{\parindent}%
\setlength{\parsep}{0pt plus 1pt}%
\addtolength{\leftmargin}{#1}%
\addtolength{\rightmargin}{#2}%
}\item }{\end{list}}
%%%%%%%%%%%
\interfootnotelinepenalty=10000
%%%evite les orphelins
\widowpenalty=10000
\clubpenalty=10000
\raggedbottom

\lccode`\'=`\'

\hyphenation{�-lec-tro-ma-gn�-ti-que po-la-ri-sa-bi-li-t�
  dif-f�-ren-tes ma-cros-co-pi-que lo-gi-que-ment dis-cr�-ti-sa-tion
  in-ci-dent u-ni-que-ment}



%%%% fin macro %%%%
%%%%%%%%%%%%%%%%%%%%%%%%%%%%%%%%%%%%%%%%%%%%%%%%%%%%
%%%%%%%%%%%%%% Debut du document %%%%%%%%%%%%%%%%%%%
%%%%%%%%%%%%%%%%%%%%%%%%%%%%%%%%%%%%%%%%%%%%%%%%%%%%
\begin{document}
\frontmatter 
\include{def}
\pagenumbering{roman}
%%%%%%%%%%%%%%%%%%%%%%%%%%%%%%%%%%%%%%%%%%%%%%%%%%%%%%
%%%%%%%%%%%%%%%%% PAGE DE GARDE  %%%%%%%%%%%%%%%%%%%%%
%%%%%%%%%%%%%%%%%%%%%%%%%%%%%%%%%%%%%%%%%%%%%%%%%%%%%%

%Une commande sembleble � \rlap ou \llap, mais centrant son argument
\def\clap#1{\hbox to 0pt{\hss #1\hss}}%
%Une commande centrant son contenu (� utiliser en mode vertical)
\def\ligne#1{%
  \hbox to \hsize{%
    \vbox{\centering #1}}}%
%Une comande qui met son premier argument � gauche, le second au 
%milieu et le dernier � droite, la premi�re ligne ce chacune de ces
%trois boites co�ncidant
\def\haut#1#2#3{%
  \hbox to \hsize{%
    \rlap{\vtop{\raggedright #1}}%
    \hss
    \clap{\vtop{\centering #2}}%
    \hss
    \llap{\vtop{\raggedleft #3}}}}%
%Idem, mais cette fois-ci, c'est la derni�re ligne
\def\bas#1#2#3{%
  \hbox to \hsize{%
    \rlap{\vbox{\raggedright #1}}%
    \hss
    \clap{\vbox{\centering #2}}%
    \hss
    \llap{\vbox{\raggedleft #3}}}}%
%La commande \maketitle
\def\maketitle{%
  \thispagestyle{empty}\vbox to \vsize{%
    \haut{}{\@blurb}{}
    \vspace{3cm}
   
    %\vfill
    \begin{center}\leavevmode
    	\normalfont
    	{\raggedleft \@author\par}%
    	%\thickhrulefill\par
    	\vspace{20mm} \hrule height 2pt 
    	{\huge\center \textbf{\@title}}%
    	\vspace{5mm} \hrule height 2pt \vspace{5mm}
	\vfill
    	\vskip 1cm
    	{\LARGE\center\textsc{}}
    	\vskip 2cm

    	
    \end{center}% 
    \vskip 1cm
    }%
  \cleardoublepage
  }

%Les commandes permettant de d�finir la date, le lieu, etc.
\def\date#1{\def\@date{#1}}
\def\author#1{\def\@author{#1}}
\def\title#1{\def\@title{#1}}
\def\location#1{\def\@location{#1}}
\def\blurb#1{\def\@blurb{#1}}
\def\email#1{\def\@email{#1}}
%Valeurs par d�faut
\date{\today}
\author{}
\title{}
\location{Marseille}
\blurb{}
\email{patrick.chaumet@fresnel.fr}
\makeatother
%
%%%%%%%%%%%%%%%%%%%%%%%%%%%%%%%%%%%%%%%%%%%%%%%%%%%%%%%%%%%%%%%%%%%%
\blurb{
\begin{center}
\parpic{
%\resizebox{160mm}{!}{\includegraphics{logofac.eps}}
}
\picskip{0}
\end{center}
 {\huge \textsc{Institut Fresnel}} 
}

\title{IF-DDA \\ \vspace{5mm} \textsc{Idiot Friendly-Discrete Dipole
    Approximation}\\ \vspace{5mm} {\Large version : 0.6.23}}

\author{\center{\LARGE \textsc{Patrick
      C. Chaumet} \\ \vspace{5mm} \textsc{Daniel Sentenac} \\
    \vspace{5mm} \textsc{Anne Sentenac}}}

\atxy(3cm,17.5cm){\resizebox{140mm}{!}{\includegraphics{schemamic.eps}}}



\email{patrick.chaumet@fresnel.fr}
\date{}


\newpage{\pagestyle{empty}\cleardoublepage}


\maketitle
\newpage{\pagestyle{empty}\cleardoublepage}
\newpage{\pagestyle{empty}\cleardoublepage}
\dominitoc 
\newpage{\pagestyle{empty}\cleardoublepage}
\tableofcontents
\clearpage{\pagestyle{empty}\cleardoublepage}
\addstarredchapter{List of figures}
\listoffigures
\clearpage{\pagestyle{empty}\cleardoublepage}
%\include{remerciements}
\mainmatter 
\pagenumbering{arabic}
\chapter{G�n�ralit�s}\label{chap1}
\markboth{\uppercase{G�n�ralit�s}}{\uppercase{G�n�ralit�s}}

\minitoc

\section{Introduction}


Ce logiciel permet de calculer la diffraction d'une onde
�lectromagn�tique par un objet tridimensionnel. Cette interaction est
prise en compte rigoureusement par la r�solution des �quations de
Maxwell, mais peut aussi le faire par des m�thodes approch�es telles
que l'approximation de Born, Rytov ou la BPM. Le code par une
interface conviviale permet de choisir des objets canoniques (sph�re,
cube,...) ainsi que des ondes incidentes pr�d�finies (onde plane,
faisceau Gaussien,...) ainsi que des objets et incidents
arbitraires. Apr�s par des menus d�roulants, il est facile d'�tudier
les sections efficaces, les forces et couples optiques, la diffraction
champ proche et champ lointain ainsi que la microscopie.


A noter qu'il existe de nombreuses m�thodes permettant d'�tudier la
diffraction d'une onde �lectromagn�tique par un objet de forme et de
permittivit� relative arbitraires. Nous n'allons par faire ici une
liste exhaustive de ces m�thodes, mais le lecteur int�ress� peut se
reporter � l'article de F. M. Kahnert qui d�taille les forces et les
faiblesses des m�thodes les plus usuelles.~\cite{Kahnert_JQSRT_03}

La m�thode que nous utilisons s'appelle la m�thode des dip�les coupl�s
(CDM) ou dip�le discret approximation (DDA). Cette m�thode, dite
volumique car le champ diffract� est obtenu � partir d'une int�grale
dont le support est le volume de l'objet consid�r�, a �t� introduite
par E. M. Purcell et C. R. Pennypacker en 1973 pour �tudier la
diffusion de la lumi�re par des grains dans le milieu
interstellaire.~\cite{Purcell_AJ_73} La DDA a �t� par la suite �tendue
� des objets en pr�sence d'un substrat plan ou dans un syst�me
multicouche, voir par exemple Ref.~[\onlinecite{Rahmani_PRA_97}]. Nous
nous sommes attach�s ces derni�res ann�es, � d'une part �tendre la DDA
� des g�om�tries plus complexes (r�seaux avec ou sans d�faut), et
d'autre part � augmenter sa pr�cision. Ces am�liorations conf�rent �
ce chapitre un c�t� un peu technique, mais elles voient leurs
applications dans les chapitres suivants.  Mais avant d'�tudier plus
en d�tails les derni�res avanc�es de la DDA, rappelons d'abord son
principe.

\section{Le principe de la DDA}\label{paprincipecdm}

Nous pr�sentons dans ce paragraphe la DDA d'une mani�re volontairement
simpliste. Soit un objet de forme et de permittivit� relative
arbitraires dans un espace homog�ne, que nous supposerons ici �tre le
vide. Cet objet est soumis � une onde �lectromagn�tique incidente de
longueur d'onde $\lambda$ ($k_0=2\pi/\lambda$). Le principe de la DDA
consiste � repr�senter l'objet en un ensemble de $N$ petits cubes
d'ar�te $a$ [par petits, nous entendons plus petits que la longueur
  d'onde dans l'objet : $a\ll \lambda/\sqrt{\varepsilon}$
  (Fig.~\ref{discretisation})].
%%%%%%%%%%%%%%%%%%%%%%%%%%%%%%%%%%%%%%%%%%%%%%%%%%%%%%%%%%%%%%%%%%%%%%
\begin{figure}
\begin{center}
\includegraphics*[draft=false,width=150mm]{discretisation.eps}
\caption{Principe de la DDA : l'objet � �tudier (� gauche) est
 discr�tis� en un ensemble de petits dip�les (� droite).}
\label{discretisation}
\end{center}
\end{figure}
%%%%%%%%%%%%%%%%%%%%%%%%%%%%%%%%%%%%%%%%%%%%%%%%%%%%%%%%%%%%%%%%%%%%%%
Chacun des petits cubes sous l'action de l'onde incidente va se
polariser, et donc acqu�rir un moment dipolaire, dont la valeur va
d�pendre du champ incident et de son interaction avec ses voisins. Le
champ local � la position d'un dip�le localis� en $\ve{r}_i$,
$\ve{E}(\ve{r}_i)$, est, en l'absence de lui-m�me, la somme de l'onde
incidente et du champ rayonn� par les $N-1$ autres dip�les :
%%%%%%%%%%%%%%%%%%%%%%%%%%%%%%%%%%%%%%%%%%%%%%%%%%%%%%%%%%%%%%%%%%%%%%
\be \label{cdms} \ve{E}(\ve{r}_i)=\ve{E}_0(\ve{r}_i)+\sum_{j=1,i\neq
j}^{N} \ve{T}(\ve{r}_i,\ve{r}_j)\alpha(\ve{r}_j)\ve{E}(\ve{r}_j). \ee
%%%%%%%%%%%%%%%%%%%%%%%%%%%%%%%%%%%%%%%%%%%%%%%%%%%%%%%%%%%%%%%%%%%%%%
$\ve{E}_0$ est le champ incident, $\ve{T}$ la susceptibilit� lin�aire
du champ en espace
homog�ne:
%%%%%%%%%%%%%%%%%%%%%%%%%%%%%%%%%%%%%%%%%%%%%%%%%%%%%%%%%%%%%%%%%%%%%%
\be \ve{T}(\ve{r}_i,\ve{r}_j)=e^{ik_0 r}
\left[\left(3\frac{\ve{r}\bigotimes\ve{r}}{r^2}- \ve{I}\right)
  \left(\frac{1}{r^3}-\frac{ik_0}{r^2}\right) +
  \left(\ve{I}-\frac{\ve{r}\bigotimes\ve{r}}{r^2}\right)
  \frac{k_0^2}{r}\right] \ee
%%%%%%%%%%%%%%%%%%%%%%%%%%%%%%%%%%%%%%%%%%%%%%%%%%%%%%%%%%%%%%%%%%%%%%
avec $\ve{I}$ la matrice unit� et $\ve{r}=\ve{r}_i-\ve{r}_j$. $\alpha$
est la polarisabilit� de chaque �l�ment de discr�tisation obtenue �
partir de la relation de Claussius-Mossotti. Notons que la
polarisabilit� $\alpha$, pour respecter le th�or�me optique, se doit
de contenir un terme dit de r�action de
rayonnement.~\cite{Draine_AJ_88} L'Eq.~(\ref{cdms}) est
vraie pour $i=1,\cdots,N$, et repr�sente donc un syst�me de $3N$
�quations lin�aires � r�soudre, les champs locaux, $\ve{E}(\ve{r}_i)$,
�tant les inconnus. Une fois le syst�me d'�quations lin�aires r�solu,
le champ diffus� par l'objet � une position $\ve{r}$ arbitraire, est
obtenu en faisant la somme de tous les champs rayonn�s par chacun des
dip�les :
%%%%%%%%%%%%%%%%%%%%%%%%%%%%%%%%%%%%%%%%%%%%%%%%%%%%%%%%%%%%%%%%%%%%%%
\be \label{cdmd} \ve{E}(\ve{r})=\sum_{j=1}^{N} \ve{T}(\ve{r},\ve{r}_j)
\alpha(\ve{r}_j) \ve{E}(\ve{r}_j). \ee
%%%%%%%%%%%%%%%%%%%%%%%%%%%%%%%%%%%%%%%%%%%%%%%%%%%%%%%%%%%%%%%%%%%%%%
Quand l'objet est en pr�sence d'un substrat plan, ou dans un
multicouche, il suffit de remplacer $\ve{T}$, par la susceptibilit�
lin�aire du champ du syst�me de r�f�rence.

Nous venons de pr�senter la DDA telle que l'ont pr�sent�e E. M.
Purcell and C. R. Pennypacker.~\cite{Purcell_AJ_73} Notons qu'une
autre m�thode tr�s proche de la DDA existe. Cette m�thode, dite
m�thode des moments, part de l'�quation int�grale de Lippman
Schwinger, est, moyennant quelques hypoth�ses, strictement identique �
la DDA. La d�monstration de l'�quivalence entre ces deux m�thodes
�tant un peu technique, elle est explicit�e dans la
Ref.~\onlinecite{Chaumet_PRE_04}.

Les avantages de la DDA sont qu'elle est applicable � des objets de
forme arbitraire, inhomog�ne (chose difficilement r�alisable dans le
cas de m�thode surfacique), et anisotrope (la polarisabilit� associ�e
aux �l�ments de discr�tisation devient alors tensorielle). La
condition d'onde sortante est automatiquement satisfaite � travers la
susceptibilit� lin�aire du champ. Notons enfin, que seul l'objet est
discr�tis�, contrairement aux m�thodes de diff�rences finies et
�l�ments finis.~\cite{Kahnert_JQSRT_03}

L'inconv�nient majeur de la DDA consiste en une croissance rapide du
temps de calcul avec l'augmentation du nombre d'�l�ments de
discr�tisation, {\it i.e.}, l'augmentation de la taille du syst�me
d'�quations lin�aires � r�soudre. Il existe des moyens pour acc�l�rer
la r�solution d'un syst�me d'�quations lin�aires de tr�s grande
taille, telle que la m�thode des gradients conjugu�s, mais malgr�
tout, des valeurs de $N>10^6$ en espace homog�ne sont difficiles �
traiter.


\section{Un mot sur le code}

Le code est pens� pour avoir une interface conviviale afin que tout le
monde puisse l'utiliser sans probl�me y compris des non
sp�cialistes. Ceci permet alors � des �tudiants de premier cycle
d'�tudier par exemple les bases de la microscopie (crit�re de
Rayleigh, notion d'ouverture num�rique,...) ou de la diffraction sans
aucun probl�me; et � des chercheurs, typiquement des biologistes,
n'ayant aucune notion des �quations de Maxwell de simuler ce que donne
un microscope (brightfield, microscope de phase, champ sombre,...) en
fonction des param�tres usuels et de l'objet. N�anmoins, ce code peut
aussi servir � des physiciens sp�cialistes de l'�lectromagn�tisme �
travers, par exemple, de calculs de forces optiques, de diffraction,
de sections efficaces, de champ proche et ceci avec de nombreux types
de faisceaux incidents et diff�rentes m�thodes de calculs du champ
�lectromagn�tique.

Le code pr�sente donc par d�faut une interface simple ou tous les
d�tails num�riques sont cach�s et o� de nombreuses options sont alors
choisies par d�faut. Mais il est facile d'acc�der � tous les
possibilit�s de code en cochant l'option interface avanc�e. Ce guide
utilisateur explique le fonctionnement de l'interface avanc�e en
commen�ant par les diff�rents approches utilis�es par le code pour
r�soudre les �quations de Maxwell.

A noter que la convivialit� du code est faite au d�triment de
l'optimisation de la RAM et le code peut donc �tre gourmand en m�moire
pour les gros objets.


\section{Comment compiler le code}

Pour faire tourner le code sur un syst�me linux il est n�cessaire
d'avoir install� les paquets suivants: qt, qt-devel, gcc-c++ et
gfortran. Noter qu'il y a trois versions du code, la premi�re en
s�quentielle qui utilise FFT singleton, la deuxi�me en parall�le et
qui utilise FFTW (Fast Fourier Transform in the west) et qui n�cessite
openmp version 4.5 minimum, et la troisi�me qui utilise en plus le
format HDF5 pour sauvegarder les donn�es dans un seul fichier
binaire. Par d�faut le code est compil� sans HDF5 et FFTW ce qui donne
donc un code avec des FFT plus lentes et qui n'est pas parall�lis� et
une �criture des datas forc�ment en ascii.

\begin{tabular}{|c|c|c|}
  \hline
  Code par d�faut & Code avec FFTW & Code avec FFTW et HDF5 \\
  \hline
  qmake-qt4 & qmake-qt4 ``CONFIG+=fftw'' & qmake-qt4 ``CONFIG+=fftw hdf5'' \\
  make & make & make \\
make install & make install & make install \\
  \hline
\end{tabular}
Puis pour lancer le code, taper, cd bin, puis ./cdm.



Sur linux la version avec FFTW n�cessite d'installer les packages FFTW
avec par exemple ``dnf install *fftw*''. Pour la version qui utilise
en plus HDF5 il faut installer en plus les packages suivant ``dnf
install hdf hdf5 hdf5-static hdf5-devel''.



Le code s'installe aussi sous windows, mais la version parall�le
n�cessite �videmment d'installer FFTW sur windows.


\section{Un mot sur les auteurs}

\begin{itemize}
\item P. C. Chaumet est professeur des universit�s � l'Institut
  Fresnel de l'Universit� d'Aix-Marseille et s'occupe du d�veloppement
  du code source fortran et de l'interface.
\item A. Sentenac est directrice de recherche au CNRS et travaille �
  l'Institut Fresnel de l'Universit� d'Aix-Marseille et participe au
  d�veloppement du code sur ce qui est li� � la diffraction champ
  lointain et la microscopie.
\item D. Sentenac de l'European Gravitational Observatory en Italie
  d�veloppe l'interface conviviale du code en C++ et Qt.
\item G. Henry � l'Institut Fresnel de l'Universit� d'Aix-Marseille
  travaille sur la partie compilation du code (Ubuntu, Fedora, etc).
\end{itemize}

\section{Un mot sur la licence}


La licence est non commerciale : ShareAlike 4.0 International 4.0
International (CC BY-NC-SA 4.0)

Vous �tes libre de:

\begin{itemize}
\item partager, copier et redistribuer.
\item adapter, changer et construire dessus.
\end{itemize}


Vous devez sous cette licence suivre les conditions suivantes:
\begin{itemize}
\item Attribution - Vous devez citer les auteurs en cas d'utilisation
  du code et indiquer si des changements ont �t� faits.
\item NonCommercial - Vous ne pouvez pas utiliser le code dans un but
  commercial.
\item ShareAlike - Si vous transformer le code ou l'utilisez dans
  d'autres codes vous devez citer les auteurs et distribuez votre
  contribution sous la m�me licence que l'original.
\end{itemize}

\section{Comment citer le code}

\begin{itemize}

\item Si seuls les fonctions de base du code sont utilis�es:

P. C. {\textsc{Chaumet}}, A. {\textsc{Sentenac}}, and
A. {\textsc{Rahmani}}, \\{\it Coupled dipole method for scatterers
  with large permittivity.}\\
Phys. Rev. E {\bf 70}, 036606 (2004).

\item Si le calcul des forces optiques est utilis� alors:

P.C. {\textsc{Chaumet}}, A. {\textsc{Rahmani}},
A. {\textsc{Sentenac}}, and G. W. {\textsc{Bryant}},\\ {\it Efficient
  computation of optical forces with the coupled dipole method.}\\
Phys. Rev. E {\bf 72}, 046708 (2005).

\item Si le calcul des couples optique est utilis�:

P. C. {\textsc{Chaumet}} and C. {\textsc{Billaudeau}},\\ {\it Coupled
  dipole method to compute optical torque: Application to a
  micropropeller.}\\
J. Appl. Phys. {\bf 101}, 023106 (2007).

\item Si le faisceau Gaussien rigoureux est utilis�:

P. C. {\textsc{Chaumet}},\\ {\it Fully vectorial highly non
  paraxial beam close to the waist.}\\
J. Opt. Soc. Am. A {\bf 23}, 3197 (2006).

\end{itemize}
   %    Generalites 
\chapter{M�thodes approch�es}\label{chapapprox}
\markboth{\uppercase{M�thodes approch�es}}{\uppercase{M�thodes approch�es}}

\minitoc

\section{Introduction}

Dans le chapitre pr�c�dent nous avons pr�sent� la DDA par une approche
simplifi� o� l'objet est un ensemble de petits dip�les rayonnant. Dans
une approche plus rigoureuse nous partons des �quations de Maxwell en
unit� Gaussienne:
%%%%%%%%%%%%%%%%%%%%%%%%%%%%%%%%%%%%%%%%%%%%%%%%%%
\be \venab \times \ve{E}^{\rm m}(\ve{r}) & = & i \frac{\omega}{c}
\ve{B}(\ve{r}) \\
\venab \times \ve{B}(\ve{r}) & = & -i \frac{\omega}{c}
\varepsilon(\ve{r}) \ve{E}^{\rm m}(\ve{r}), \ee
%%%%%%%%%%%%%%%%%%%%%%%%%%%%%%%%%%%%%%%%%%%%%%%%%%
o� $\varepsilon(\ve{r})$ est la permittivit� relative de l'objet et
$\ve{E}^{\rm m}$ le champ total dans l'objet. En dehors de l'objet
nous avons la m�me relation avec $\varepsilon=1$. Ceci nous donne
l'�quation de propagation suivante pour le champ �lectrique:
%%%%%%%%%%%%%%%%%%%%%%%%%%%%%%%%%%%%%%%%%%%%%%%%%%
\be \venab \times ( \venab \times \ve{E}^{\rm m}(\ve{r}) ) & = &
\varepsilon(\ve{r}) k_0^2 \ve{E}^{\rm m}(\ve{r}), \ee 
%%%%%%%%%%%%%%%%%%%%%%%%%%%%%%%%%%%%%%%%%%%%%%%%%%
avec $k_0=\omega^2/c^2$. En utilisant la relation
$\varepsilon=1+4\pi \chi$ avec $\chi$ la susceptibilit� lin�aire
�lectrique nous avons:
%%%%%%%%%%%%%%%%%%%%%%%%%%%%%%%%%%%%%%%%%%%%%%%%%%
\be \venab \times ( \venab \times \ve{E}^{\rm m}(\ve{r}) ) -k_0^2
\ve{E}^{\rm m}(\ve{r}) & = & 4\pi \chi(\ve{r}) k_0^2 \ve{E}^{\rm
  m}(\ve{r}) . \label{champref}\ee
%%%%%%%%%%%%%%%%%%%%%%%%%%%%%%%%%%%%%%%%%%%%%%%%%%
La solution de cette �quation sans second membre est le champ incident
et correspond donc au milieu de r�f�rence, c'est � dire le milieu en
l'absence de l'objet �tudi� ($\chi=0$), dans notre cas le vide.  Pour
r�soudre cette �quation avec second membre on cherche la fonction de
Green solution de
%%%%%%%%%%%%%%%%%%%%%%%%%%%%%%%%%%%%%%%%%%%%%%%%%%
\be \venab \times ( \venab \times \ve{T}(\ve{r},\ve{r}') ) -k_0^2
\ve{T}(\ve{r},\ve{r}') & = & 4\pi k_0^2 \ve{I}
\delta(\ve{r}-\ve{r}'). \ee
%%%%%%%%%%%%%%%%%%%%%%%%%%%%%%%%%%%%%%%%%%%%%%%%%%
La solution finale est donc:
%%%%%%%%%%%%%%%%%%%%%%%%%%%%%%%%%%%%%%%%%%%%%%%%%%
\be\ve{E}^{\rm m}(\ve{r}) = \ve{E}_0(\ve{r}) +\int_{\Omega}
\ve{T}(\ve{r},\ve{r}') \chi(\ve{r}') \ve{E}^{\rm m}(\ve{r}') {\rm d}
\ve{r}',\ee
%%%%%%%%%%%%%%%%%%%%%%%%%%%%%%%%%%%%%%%%%%%%%%%%%%
avec $\ve{E}^0$ le champ incident solution de l'Eq.~(\ref{champref})
sans second membre et $\Omega$ le volume correspondant au support de
l'objet �tudi�. Quand on r�sout l'�quation dans l'objet, le champ
total correspond donc au champ macroscopique dans l'objet. Pour
r�soudre cette �quation on discr�tise l'objet en un ensemble de $N$
�l�ments de forme cubique ayant une ar�te de taille $d$ et l'int�grale
$\Omega$ sur l'objet est donc d�compos�e en une somme d'int�grale sur
chacun des �l�ments de discr�tisation de volume $V_j=d^3$:
%%%%%%%%%%%%%%%%%%%%%%%%%%%%%%%%%%%%%%%%%%%%%%%%%%
\be\ve{E}^{\rm m}(\ve{r}_i) = \ve{E}^0(\ve{r}_i) +\sum_{j=1}^{N}
\int_{V_j} \ve{T}(\ve{r}_i,\ve{r}') \chi(\ve{r}') \ve{E}^{\rm
  m}(\ve{r}') {\rm d} \ve{r}',\ee
%%%%%%%%%%%%%%%%%%%%%%%%%%%%%%%%%%%%%%%%%%%%%%%%%%
En supposant le champ, la fonction Green et la permittivit� constants
dans la maille, nous obtenons:
%%%%%%%%%%%%%%%%%%%%%%%%%%%%%%%%%%%%%%%%%%%%%%%%%%
\be\ve{E}^{\rm m}(\ve{r}_i) = \ve{E}^0(\ve{r}_i) +\sum_{j=1,}^N
\ve{T}(\ve{r}_i,\ve{r}_j) \chi(\ve{r}_j) \ve{E}^{\rm m}(\ve{r}_j)
d^3.\ee
%%%%%%%%%%%%%%%%%%%%%%%%%%%%%%%%%%%%%%%%%%%%%%%%%%
En utilisant, en premi�re approximation (c'est � dire que la r�action
de rayonnement est n�glig�e, mais la prendre en compte ne changerait
pas les raisonnements qui suivent), le fait que que
$\int_{V_i}\ve{T}(\ve{r}_i,\ve{r}') {\rm d} \ve{r}'= -4\pi/3 $, voir
Ref.~\onlinecite{Yaghjian_PIEEE_80}) pour plus de d�tails, nous avons:
%%%%%%%%%%%%%%%%%%%%%%%%%%%%%%%%%%%%%%%%%%%%%%%%%%
\be\ve{E}^{\rm m}(\ve{r}_i) = \ve{E}^0(\ve{r}_i) +\sum_{j=1,i\neq j}^N
\ve{T}(\ve{r}_i,\ve{r}_j) \chi(\ve{r}_j) d^3 \ve{E}^{\rm
  m}(\ve{r}_j)-\frac{4\pi}{3}\chi(\ve{r}_i) \ve{E}^{\rm m}(\ve{r}_i)
.\ee
%%%%%%%%%%%%%%%%%%%%%%%%%%%%%%%%%%%%%%%%%%%%%%%%%%
En passant toutes les d�pendances en $i$ � gauche de la relation nous
avons au final:
%%%%%%%%%%%%%%%%%%%%%%%%%%%%%%%%%%%%%%%%%%%%%%%%%%
\be\ve{E}(\ve{r}_i) & = & \ve{E}^0(\ve{r}_i) +\sum_{j=1,i\neq j}^N
\ve{T}(\ve{r}_i,\ve{r}_j) \alpha_{\rm CM}(\ve{r}_j) \ve{E}(\ve{r}_j) \\
{\rm avec} \phantom{000} \ve{E}(\ve{r}_i) & = &
\frac{\varepsilon(\ve{r}_i)+2}{3}
\ve{E}^{\rm m}(\ve{r}_i) \\
\alpha_{\rm CM}(\ve{r}_j) & = & \frac{3}{4\pi} d^3
\frac{\varepsilon(\ve{r}_i)-1}{\varepsilon(\ve{r}_i)+2} .\ee
%%%%%%%%%%%%%%%%%%%%%%%%%%%%%%%%%%%%%%%%%%%%%%%%%%
Le champ $\ve{E}(\ve{r}_i)$ est le champ local, c'est � dire que c'est
le champ dans la maille $i$ en l'absence de la maille elle m�me. En
�crivant cette �quation pour toutes les valeurs de $i$ nous avons un
syst�me d'�quations lin�aires que nous pouvons �crire symboliquement
comme:
%%%%%%%%%%%%%%%%%%%%%%%%%%%%%%%%%%%%%%%%%%%%%%%%%%
\be \ve{E} = \ve{E}^0 + \ve{A} \ve{D}_\alpha \ve{E},\ee
%%%%%%%%%%%%%%%%%%%%%%%%%%%%%%%%%%%%%%%%%%%%%%%%%%
avec $\ve{A}$ qui contient toutes les fonctions de Green et
$\ve{D}_\alpha$ une matrice diagonale qui contient toutes les
polarisabilit�s de chaque �l�ment de discr�tisation. Nous d�taillons
au chapitre suivant comment r�soudre rigoureusement ce syst�me
d'�quation lin�aire, mais dans ce pr�sent chapitre nous d�taillons les
diff�rentes approches possibles pour �viter la r�solution du syst�me
qui est tr�s gourmande en temps de calcul.

A noter que le champ diffract� par l'objet en dehors du support de
l'objet s'�crit simplement comme:
%%%%%%%%%%%%%%%%%%%%%%%%%%%%%%%%%%%%%%%%%%%%%%%%%%
\be\ve{E}^{\rm d}(\ve{r}) & = & \sum_{j=1}^N \ve{T}(\ve{r},\ve{r}_j)
\alpha(\ve{r}_j) \ve{E}(\ve{r}_j). \ee
%%%%%%%%%%%%%%%%%%%%%%%%%%%%%%%%%%%%%%%%%%%%%%%%%%



\section{Les diff�rentes m�thodes approch�es utilis�es dans le code}


\subsection{Born}


Une approximation simple est l'approximation de Born, c'est � dire que
le champ macroscopique dans l'objet est le champ incident. Nous avons
donc :
%%%%%%%%%%%%%%%%%%%%%%%%%%%%%%%%%%%%%%%%%%%%%%%%%%
\be \ve{E}^{\rm m}(\ve{r}_i) = \ve{E}^0(\ve{r}_i), \ee
%%%%%%%%%%%%%%%%%%%%%%%%%%%%%%%%%%%%%%%%%%%%%%%%%%
pour tous les �l�ments de discr�tisation. Apr�s il suffit de faire
propager le champ. Il est �vident que cette approximation tient si le
contraste et la taille de l'objet sont petits.


\subsection{Born renormalis�}

Nous pouvons faire l'hypoth�se � l'identique mais sur le champ local,
c'est � dire que :
%%%%%%%%%%%%%%%%%%%%%%%%%%%%%%%%%%%%%%%%%%%%%%%%%%
\be \ve{E}(\ve{r}_i) = \ve{E}^0(\ve{r}_i). \ee
%%%%%%%%%%%%%%%%%%%%%%%%%%%%%%%%%%%%%%%%%%%%%%%%%%
En consid�rant la relation entre le champ local et le champ
macroscopique nous avons alors:
%%%%%%%%%%%%%%%%%%%%%%%%%%%%%%%%%%%%%%%%%%%%%%%%%%
\be \ve{E}^{\rm m}(\ve{r}_i) = \frac{3}{\varepsilon(\ve{r}_i)+2}
\ve{E}^0(\ve{r}_i). \ee
%%%%%%%%%%%%%%%%%%%%%%%%%%%%%%%%%%%%%%%%%%%%%%%%%%
La phase est la m�me que dans le cas de l'approximation de Born mais
l'amplitude est chang�e. Cette approximation est meilleure pour des
permittivit�s plus fortes car fait une correction sur l'amplitude du
champ macroscopique, nous avons appel� cette approximation Born
renormalis�.


\subsection{Born � l'ordre 1}

Sans r�soudre le syst�me d'�quations lin�aires on peut faire un Born
renormalis� � l'ordre 1, c'est � dire que l'on effectue:
%%%%%%%%%%%%%%%%%%%%%%%%%%%%%%%%%%%%%%%%%%%%%%%%%%
\be\ve{E}(\ve{r}_i) & = & \ve{E}^0(\ve{r}_i) +\sum_{j=1,i\neq j}^N
\ve{T}(\ve{r}_i,\ve{r}_j) \alpha(\ve{r}_j) \ve{E}^0(\ve{r}_j). \ee
%%%%%%%%%%%%%%%%%%%%%%%%%%%%%%%%%%%%%%%%%%%%%%%%%%
Ceci permet de prendre en compte un peu la variation du champ dans
l'objet et permet de traiter des objets plus grands mais toujours avec
un contraste faible. Il est possible de d�velopper Born � des ordres
sup�rieurs mais quand le contraste devient fort la s�rie ne converge
plus...


\subsection{Rytov}

L'approximation de Rytov consiste � changer la phase du champ
incident. Pour ce faire nous calculons :
%%%%%%%%%%%%%%%%%%%%%%%%%%%%%%%%%%%%%%%%%%%%%%%%%%
\be\ve{E}^{\rm d}(\ve{r}_i) & = & \sum_{j=1}^N
\ve{T}(\ve{r}_i,\ve{r}_j) \chi(\ve{r}_j) \ve{E}^0(\ve{r}_j), \ee
%%%%%%%%%%%%%%%%%%%%%%%%%%%%%%%%%%%%%%%%%%%%%%%%%%
toujours avec $\ve{T}(\ve{r}_i,\ve{r}_j)=-\frac{4\pi}{3 d^3}$, puis le
champ macroscopique dans l'objet est estim� par:
%%%%%%%%%%%%%%%%%%%%%%%%%%%%%%%%%%%%%%%%%%%%%%%%%%
\be E_\beta^{\rm m}(\ve{r}_i) & = & E_\beta^0(\ve{r}_i) e^{E^{\rm
    d}_\beta(\ve{r}_i)/E^0_\beta(\ve{r}_i)}, \ee
%%%%%%%%%%%%%%%%%%%%%%%%%%%%%%%%%%%%%%%%%%%%%%%%%%
avec $\beta=x,y,z$.  Lorsque la composante du champ incidente est
nulle alors $E_\beta^{\rm m}$ est aussi nulle.  Cette approximation
permet de traiter des objets grands par rapport � la longueur d'onde,
mais toujours avec un contraste faible. Comme pour Born cela demande
de faire un produit matrice vecteur. A noter que l'amplitude utilis�e
c'est celle du champ incident.

\subsection{Rytov renormalis�}


L'approximation de Rytov renormalis�e consiste � faire la m�me chose
que Rytov mais en travaillant sur le champ local. Nous avons alors:
%%%%%%%%%%%%%%%%%%%%%%%%%%%%%%%%%%%%%%%%%%%%%%%%%%
\be\ve{E}^{\rm d}(\ve{r}_i) & = & \sum_{j=1,i\neq j}^N
\ve{T}(\ve{r}_i,\ve{r}_j) \alpha(\ve{r}_j) \ve{E}^0(\ve{r}_j), \ee
%%%%%%%%%%%%%%%%%%%%%%%%%%%%%%%%%%%%%%%%%%%%%%%%%%
puis le champ local dans l'objet est estim� par:
%%%%%%%%%%%%%%%%%%%%%%%%%%%%%%%%%%%%%%%%%%%%%%%%%%
\be E_\beta(\ve{r}_i) & = & E_\beta^0(\ve{r}_i) e^{E_\beta^{\rm
    d}(\ve{r}_i)/E_\beta^0(\ve{r}_i)}. \ee
%%%%%%%%%%%%%%%%%%%%%%%%%%%%%%%%%%%%%%%%%%%%%%%%%%
Cela permet d'avoir un contraste un peu plus fort.


\subsection{M�thode de propagation du faisceau (BPM)}


Cette m�thode est compl�tement diff�rentes des pr�c�dentes car elle ne
fait pas du tout appelle � la r�solution d'un syst�me d'�quations
lin�aires, mais fait la propagation de l'onde en tenant compte de
l'indice du milieu. Elle s'applique donc dans le cas d'objet pouvant
�tre tr�s grand mais pr�sentant un contraste faible avec des chocs
d'indices tr�s faibles. Pour plus de d�tails sur la m�thode voir
Ref.~\onlinecite{Kamilov_IEEE_16} mais au final le champ dans l'objet
s'�crit comme:
%%%%%%%%%%%%%%%%%%%%%%%%%%%%%%%%%%%%%%%%%%%%%%%%%
\be \ve{E}^{\rm m}(x,y,z+d)= e^{i k_0 n(x,y,z+d) d } {\cal
  F}^{-1}\left[ {\cal F} [\ve{E}^{\rm m}(x,y,z)] e^{-i(k_0-k_z) d}
\right], \ee
%%%%%%%%%%%%%%%%%%%%%%%%%%%%%%%%%%%%%%%%%%%%%%%%%
o� le calcul du champ � la position $(x,y,z+d)$ se fait avec la valeur
de l'indice optique � la m�me position et de la valeur du champ au
plan pr�c�dent $z$. On propage ainsi de maille en maille dans la
direction $z$ pour conna�tre le champ dans tout l'objet. Il est clair
qu'avec cette m�thode le champ ne se propage que dans la direction des
$z$ positifs, il n'y a jamais de r�flexion vers l'arri�re.  A noter
que la FFT utilis�e � la taille d�finie par le menu d�roulant sur la
FFT et surtout pas la taille de l'objet qui pourrait �tre trop petite
et manqu�e de pr�cision. Le champ diffract� est calcul� comme
d'habitude, ce qui permet d'�tre bien meilleur que d'utiliser
l'int�grale de Kirchhoff comme c'est souvent fait.

\subsection{M�thode de propagation du faisceau renormalis�e (BPM)}

Nous pouvons faire la m�me hypoth�se que pr�c�demment mais sur le
champ local, soit:
%%%%%%%%%%%%%%%%%%%%%%%%%%%%%%%%%%%%%%%%%%%%%%%%%
\be \ve{E}(x,y,z+d)= e^{i k_0 n(x,y,z+d) d } {\cal F}^{-1}\left[ {\cal
    F} [\ve{E}(x,y,z)] e^{-i(k_0-k_z) d} \right]. \ee
%%%%%%%%%%%%%%%%%%%%%%%%%%%%%%%%%%%%%%%%%%%%%%%%%
\include{chappola}   %    Gestion des configurations
\include{chap2}   %    Gestion des configurations
\include{chap3}   %    Illumination
\include{chap4}   %    D�finition de l'objet
\include{chap5}   %    Etude
\include{chap6}   %    Repr�sentation des r�sultats 
\include{chap7}   %    Output file for matlab
\chapter{Quleques exemples}\label{chaptest}
\markboth{\uppercase{Fichiers de test}}{\uppercase{Fichiers de test}}

\minitoc

\section{Introduction}

Dans bin/tests est dispos� un fichier options.db3. Si on le copie un
directory en dessous ``cp options.db3 ../.'', quand on lance le code
apr�s un load il appara�t quatre configurations tests qui permettent de
voir toutes les options en action.

\section{Test1}

Le but du test1 est de tester un cas simple et de nombreuses options
du code afin de les valider.  La Fig.~\ref{test1conf} montre les
options de la configuration choisie.


%%%%%%%%%%%%%%%%%%%%%%%%%%%%%%%%%%%%%%%%%%%%%%%
\begin{figure}[H]
\begin{center}
  \includegraphics*[width=15.0cm,draft=false]{test1conf.eps}
\end{center}
\caption{Test1: configuration choisie.}
\label{test1conf}
\end{figure}
%%%%%%%%%%%%%%%%%%%%%%%%%%%%%%%%%%%%%%%%%%%%%%%

Les Figures suivantes montrent les r�sultats obtenus.  Les trac�s sont
effectu�s avec Matlab et ce sont directement les fichiers eps tir�s du
script ifdda.m qui sont utilis�s, mais ceux-ci peuvent bien s�r �tre 
r�alis�s avec l'interface graphique int�gr�e. L'avantage de matlab dans
ce cas est de donner toutes les figures d'un seul coup.
%%%%%%%%%%%%%%%%%%%%%%%%%%%%%%%%%%%%%%%%%%%%%%%
\begin{figure}[H]
\begin{center}
  \includegraphics*[width=15.0cm,draft=false]{test1local.eps}
\end{center}
\caption{Module du champ local dans le plan $(x,y)$.}
%\label{test1res1}
\end{figure}
%%%%%%%%%%%%%%%%%%%%%%%%%%%%%%%%%%%%%%%%%%%%%%%
%%%%%%%%%%%%%%%%%%%%%%%%%%%%%%%%%%%%%%%%%%%%%%%
\begin{figure}[H]
\begin{center}
  \includegraphics*[width=15.0cm,draft=false]{test1macro.eps}
\end{center}
\caption{Module du champ macroscopique dans le plan $(x,y)$.}
%\label{test1res1}
\end{figure}
%%%%%%%%%%%%%%%%%%%%%%%%%%%%%%%%%%%%%%%%%%%%%%%
Le champ incident �tant polaris� suivant la composante $y$ (TE), il
est clair que la composante $y$ du champ � l'int�rieur de la sph�re
est la plus forte.

%%%%%%%%%%%%%%%%%%%%%%%%%%%%%%%%%%%%%%%%%%%%%%%
\begin{figure}[H]
\begin{center}
  \includegraphics*[width=15.0cm,draft=false]{test1poynting2d.eps}
\end{center}
\caption{Champ rayonn� par l'objet.}
%\label{test1res1}
\end{figure}
%%%%%%%%%%%%%%%%%%%%%%%%%%%%%%%%%%%%%%%%%%%%%%%
%%%%%%%%%%%%%%%%%%%%%%%%%%%%%%%%%%%%%%%%%%%%%%%
\begin{figure}[H]
\begin{center}
\begin{tabular}{cc}
  \includegraphics*[width=7.0cm,draft=false]{test1force2d.eps}
&  \includegraphics*[width=9.0cm,draft=false]{test1force3d.eps}
\end{tabular}

\end{center}
\caption{Force optique dans la plan $(x,y)$ et en trois D.}
%\label{test1res2}
\end{figure}
%%%%%%%%%%%%%%%%%%%%%%%%%%%%%%%%%%%%%%%%%%%%%%%
%%%%%%%%%%%%%%%%%%%%%%%%%%%%%%%%%%%%%%%%%%%%%%%
\begin{figure}[H]
\begin{center}
\begin{tabular}{cc}
  \includegraphics*[width=7.0cm,draft=false]{test1torque2d.eps}
&  \includegraphics*[width=9.0cm,draft=false]{test1torque3d.eps}
\end{tabular}

\end{center}
\caption{Couple optique dans la plan $(x,y)$ et en trois D.}
%\label{test1res2}
\end{figure}
%%%%%%%%%%%%%%%%%%%%%%%%%%%%%%%%%%%%%%%%%%%%%%%

%%%%%%%%%%%%%%%%%%%%%%%%%%%%%%%%%%%%%%%%%%%%%%%
\begin{figure}[H]
\begin{center}
\begin{tabular}{ccc}
  \includegraphics*[width=5.0cm,draft=false]{test1fourier.eps}
& \includegraphics*[width=5.0cm,draft=false]{test1image.eps}
&  \includegraphics*[width=5.0cm,draft=false]{test1imageinc.eps}

\end{tabular}

\end{center}
\caption{Microscopie en tramsmission: Module du champ diffract� dans
  le domaine de Fourier (gauche), module du champ diffract� dans le
  plan image (milieu), et module du champ total dans le plan image
  (droite).}
\end{figure}
%%%%%%%%%%%%%%%%%%%%%%%%%%%%%%%%%%%%%%%%%%%%%%%

\section{Test2}

Le but du test2 est de tester un cas simple et de nombreuses options
du code afin de les valider.  La Fig.~\ref{test2conf} montre les
options de la configuration choisie. L'�clairement est fait par deux
ondes planes qui interf�rent.


%%%%%%%%%%%%%%%%%%%%%%%%%%%%%%%%%%%%%%%%%%%%%%%
\begin{figure}[H]
\begin{center}
  \includegraphics*[width=15.0cm,draft=false]{test2conf.eps}
\end{center}
\caption{Test2: configuration choisie.}
\label{test2conf}
\end{figure}
%%%%%%%%%%%%%%%%%%%%%%%%%%%%%%%%%%%%%%%%%%%%%%%


%%%%%%%%%%%%%%%%%%%%%%%%%%%%%%%%%%%%%%%%%%%%%%%
\begin{figure}[H]
\begin{center}
\begin{tabular}{ccc}
  \includegraphics*[width=7.0cm,draft=false]{test2dipolepos.eps}
& \includegraphics*[width=7.0cm,draft=false]{test2epsilon.eps}
\end{tabular}

\end{center}
\caption{Repr�sentation tridimensionnelle de l'objet (gauche) et carte
  de permittivit� dans le plan $(x,y)$ (droite).}
\end{figure}
%%%%%%%%%%%%%%%%%%%%%%%%%%%%%%%%%%%%%%%%%%%%%%%


Les Figures suivantes montrent les r�sultats obtenus.
%%%%%%%%%%%%%%%%%%%%%%%%%%%%%%%%%%%%%%%%%%%%%%%
\begin{figure}[H]
\begin{center}
  \includegraphics*[width=15.0cm,draft=false]{test2incident.eps}
\end{center}
\caption{Module du champ incident dans le plan $(x,y)$.}
%\label{test1res1}
\end{figure}
%%%%%%%%%%%%%%%%%%%%%%%%%%%%%%%%%%%%%%%%%%%%%%%
%%%%%%%%%%%%%%%%%%%%%%%%%%%%%%%%%%%%%%%%%%%%%%%
\begin{figure}[H]
\begin{center}
  \includegraphics*[width=15.0cm,draft=false]{test2local.eps}
\end{center}
\caption{Module du champ local dans le plan $(x,y)$.}
%\label{test1res1}
\end{figure}
%%%%%%%%%%%%%%%%%%%%%%%%%%%%%%%%%%%%%%%%%%%%%%%
%%%%%%%%%%%%%%%%%%%%%%%%%%%%%%%%%%%%%%%%%%%%%%%
\begin{figure}[H]
\begin{center}
  \includegraphics*[width=15.0cm,draft=false]{test2macro.eps}
\end{center}
\caption{Module du champ macroscopique dans le plan $(x,y)$.}
%\label{test1res1}
\end{figure}
%%%%%%%%%%%%%%%%%%%%%%%%%%%%%%%%%%%%%%%%%%%%%%%

%%%%%%%%%%%%%%%%%%%%%%%%%%%%%%%%%%%%%%%%%%%%%%%
\begin{figure}[H]
\begin{center}
  \includegraphics*[width=15.0cm,draft=false]{test2poynting2d.eps}
\end{center}
\caption{Champ rayonn� par l'objet.}
%\label{test1res1}
\end{figure}
%%%%%%%%%%%%%%%%%%%%%%%%%%%%%%%%%%%%%%%%%%%%%%%


%%%%%%%%%%%%%%%%%%%%%%%%%%%%%%%%%%%%%%%%%%%%%%%
\begin{figure}[H]
\begin{center}
\begin{tabular}{ccc}
  \includegraphics*[width=7.0cm,draft=false]{test2fourier.eps}
& \includegraphics*[width=7.0cm,draft=false]{test2image.eps}
\end{tabular}

\end{center}
\caption{Microscopie en r�flexion: Module du champ diffract� dans le
  domaine de Fourier (gauche), module du champ diffract� dans le plan
  image (droite).}
\end{figure}
%%%%%%%%%%%%%%%%%%%%%%%%%%%%%%%%%%%%%%%%%%%%%%%




\section{Test3}

Le but du test3 est de tester la microscopie en champ sombre et champ
brillant en transmission. On �tudie une sph�re de 500~nm de rayon et
de permittivit� 1.5.


%%%%%%%%%%%%%%%%%%%%%%%%%%%%%%%%%%%%%%%%%%%%%%%
\begin{figure}[H]
\begin{center}
  \includegraphics*[width=15.0cm,draft=false]{test3conf.eps}
\end{center}
\caption{Test3: configuration choisie.}
\label{test3conf}
\end{figure}
%%%%%%%%%%%%%%%%%%%%%%%%%%%%%%%%%%%%%%%%%%%%%%%


%%%%%%%%%%%%%%%%%%%%%%%%%%%%%%%%%%%%%%%%%%%%%%%
\begin{figure}[H]
\begin{center}
\begin{tabular}{ccc}
 \includegraphics*[width=5.0cm,draft=false]{test3angleincbf.eps}
&  \includegraphics*[width=5.0cm,draft=false]{test3imagewf.eps}
& \includegraphics*[width=5.0cm,draft=false]{test3imageincwf.eps}
\end{tabular}

\end{center}
\caption{Microscopie en transmission: incident pris pour cr�er l'image
  (gauche). Module du champ diffract� dans le plan image (milieu),
  module du champ total dans le plan image (droite).}
\end{figure}
%%%%%%%%%%%%%%%%%%%%%%%%%%%%%%%%%%%%%%%%%%%%%%%


\section{Test4}

Le but du test4 est de tester la microscopie en champ sombre et champ
brillant entransmission. On �tudie une sph�re de 500~nm de rayon est
de permittivt� 1.5.


%%%%%%%%%%%%%%%%%%%%%%%%%%%%%%%%%%%%%%%%%%%%%%%
\begin{figure}[H]
\begin{center}
  \includegraphics*[width=15.0cm,draft=false]{test4conf.eps}
\end{center}
\caption{Test4: configuration choisie.}
\label{test4conf}
\end{figure}
%%%%%%%%%%%%%%%%%%%%%%%%%%%%%%%%%%%%%%%%%%%%%%%


%%%%%%%%%%%%%%%%%%%%%%%%%%%%%%%%%%%%%%%%%%%%%%%
\begin{figure}[H]
\begin{center}
\begin{tabular}{cc}
 \includegraphics*[width=7.0cm,draft=false]{test4angleincdf.eps}
&  \includegraphics*[width=7.0cm,draft=false]{test4imagewf.eps}
\end{tabular}

\end{center}
\caption{Microscopie en r�flexion: incident pris pour cr�er l'image
  (gauche). Module du champ diffract� dans le plan image (droite).}
\end{figure}
%%%%%%%%%%%%%%%%%%%%%%%%%%%%%%%%%%%%%%%%%%%%%%%
   %    test

\addstarredchapter{Bibliographie}
\markboth{\uppercase{Bibliographie}}{\uppercase{Bibliographie}}

\bibliographystyle{apsrev} 
\bibliography{bibliographie}



\end{document}
