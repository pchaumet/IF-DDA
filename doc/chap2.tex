\chapter{Gestion des configurations}\label{chap2}
\markboth{\uppercase{Gestion des configurations}}{\uppercase{Gestion
    des configurations}}

\minitoc

\section{Introduction}


Le Code se lance par ./cdm dans le dossier bin pour une configuration
linux. Celui-ci a �t� fait pour �tre le plus convivial possible et
n�cessite donc peu d'explication pour son utilisation. N�anmoins
certaines conventions ont �t� prises et demandent � �tre explicit�es.

\section{Cr�ation et sauvegarde d'une nouvelle configuration}

Pour d�marrer un nouveau calcul, aller sur l'onglet {\it calculation}
et {\it New}. Une nouvelle configuration s'affiche avec des valeurs
par d�faut. Une fois la nouvelle configuration choisie, pour la sauver
il faut choisir de nouveau l'onglet {\it Calculation} et {\it
  Save}. On choisit alors le nom de la configuration et on peut
ajouter une courte description du calcul fait.

Une autre mani�re de sauvegarder une configuration, c'est de cliquer
directement sur le panneau de la configuration {\it Save
  configuration}. Il appara�t alors deux champs, un pour le nom de la
configuration et le deuxi�me pour sa description.

\section{Gestion des configurations}

Pour g�rer toutes les configurations choisies, il faut aller sur
l'onglet {\it Calculation} et {\it Load}. Il appara�t alors une
nouvelle fen�tre avec toutes les configurations sauv�es. Pour chaque
configuration il y a une courte description que l'utilisateur a
rentr�, la date, o� le fichier configuration a �t� sauv�, puis les
caract�ristiques principales de la configuration (longueur d'onde,
puissance, col du faisceau, objet, mat�riau, discr�tisation et
tol�rance de la m�thode it�rative).  Il suffit de cliquer sur une
configuration et de cliquer sur {\it load} pour charger une
configuration.

Le bouton {\it delete} sert a supprimer une configuration sauvegard�e
et le bouton {\it export} permet d'exporter dans un fichier (nom de la
configuration.opt) toutes les caract�ristiques de la configuration.

A noter qu'en double cliquant sur la ligne, on peut modifier le champ
description.
