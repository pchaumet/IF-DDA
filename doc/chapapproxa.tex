\chapter{Approximated method}\label{chapapprox}
\markboth{\uppercase{Approximated method}}{\uppercase{Approximated method}}

\minitoc

\section{Introduction}

In the previous chapter we have presented the DDA in a simple way
where the object under study is a set of radiating dipole. In an
approach more rigorous, with the Maxwell's equation, we get in
Gaussian unit:
%%%%%%%%%%%%%%%%%%%%%%%%%%%%%%%%%%%%%%%%%%%%%%%%%%
\be \venab \times \ve{E}^{\rm m}(\ve{r}) & = & i \frac{\omega}{c}
\ve{B}(\ve{r}) \\
\venab \times \ve{B}(\ve{r}) & = & -i \frac{\omega}{c}
\varepsilon(\ve{r}) \ve{E}^{\rm m}(\ve{r}), \ee
%%%%%%%%%%%%%%%%%%%%%%%%%%%%%%%%%%%%%%%%%%%%%%%%%%
where $\varepsilon(\ve{r})$ denotes the relative permittivity of the
object and $\ve{E}^{\rm m}$ the macroscopic field inside the object,
then we get
%%%%%%%%%%%%%%%%%%%%%%%%%%%%%%%%%%%%%%%%%%%%%%%%%%
\be \venab \times ( \venab \times \ve{E}^{\rm m}(\ve{r}) ) & = &
\varepsilon(\ve{r}) k_0^2 \ve{E}^{\rm m}(\ve{r}), \ee 
%%%%%%%%%%%%%%%%%%%%%%%%%%%%%%%%%%%%%%%%%%%%%%%%%%
with $k_0=\omega^2/c^2$. Using the relationship
$\varepsilon=1+4\pi \chi$, where $\chi$ denotes the linear field
susceptibility, we have:
%%%%%%%%%%%%%%%%%%%%%%%%%%%%%%%%%%%%%%%%%%%%%%%%%
\be \venab \times ( \venab \times \ve{E}^{\rm m}(\ve{r}) ) -k_0^2
\ve{E}^{\rm m}(\ve{r}) & = & 4\pi \chi(\ve{r}) k_0^2 \ve{E}^{\rm
  m}(\ve{r}) . \label{champref}\ee
%%%%%%%%%%%%%%%%%%%%%%%%%%%%%%%%%%%%%%%%%%%%%%%%%%
To solve this equation one needs the Green function defined as:
%%%%%%%%%%%%%%%%%%%%%%%%%%%%%%%%%%%%%%%%%%%%%%%%%%
\be \venab \times ( \venab \times \ve{T}(\ve{r},\ve{r}') ) -k_0^2
\ve{T}(\ve{r},\ve{r}') & = & 4\pi k_0^2 \ve{I}
\delta(\ve{r}-\ve{r}'), \ee
%%%%%%%%%%%%%%%%%%%%%%%%%%%%%%%%%%%%%%%%%%%%%%%%%%
and the solution of Eq.~(\ref{champref}) reads:
%%%%%%%%%%%%%%%%%%%%%%%%%%%%%%%%%%%%%%%%%%%%%%%%%%
\be\ve{E}^{\rm m}(\ve{r}) = \ve{E}_0(\ve{r}) +\int_{\Omega}
\ve{T}(\ve{r},\ve{r}') \chi(\ve{r}') \ve{E}^{\rm m}(\ve{r}') {\rm d}
\ve{r}',\ee
%%%%%%%%%%%%%%%%%%%%%%%%%%%%%%%%%%%%%%%%%%%%%%%%%%
where $\ve{E}^0$ is the incident field and $\Omega$ the support of the
object under study. When we solve Eq.~(\ref{champref}) the field
$\ve{E}^{\rm m}$ corresponds to macroscopic field inside the object.
To solve Eq.~(\ref{champref}) we discretize the object in a set of $N$
subunits with a cubic meshsize $d$, then the integral equation becomes
the sum of $N$ integrals:
%%%%%%%%%%%%%%%%%%%%%%%%%%%%%%%%%%%%%%%%%%%%%%%%%%
\be\ve{E}^{\rm m}(\ve{r}_i) = \ve{E}^0(\ve{r}_i) +\sum_{j=1}^{N}
\int_{V_j} \ve{T}(\ve{r}_i,\ve{r}') \chi(\ve{r}') \ve{E}^{\rm
  m}(\ve{r}') {\rm d} \ve{r}',\ee
%%%%%%%%%%%%%%%%%%%%%%%%%%%%%%%%%%%%%%%%%%%%%%%%%%
with $V_j=d^3$.  Assuming the field, the Green function and the
susceptibility constant over a subunit we get:
%%%%%%%%%%%%%%%%%%%%%%%%%%%%%%%%%%%%%%%%%%%%%%%%%%
\be\ve{E}^{\rm m}(\ve{r}_i) = \ve{E}^0(\ve{r}_i) +\sum_{j=1}^N
\ve{T}(\ve{r}_i,\ve{r}_j) \chi(\ve{r}_j) \ve{E}^{\rm m}(\ve{r}_j)
d^3.\ee
%%%%%%%%%%%%%%%%%%%%%%%%%%%%%%%%%%%%%%%%%%%%%%%%%%
Using, in first approximation (the radiative reaction term neglected)
$\int_{V_i}\ve{T}(\ve{r}_i,\ve{r}') {\rm d} \ve{r}'= -4\pi/3
$~\cite{Yaghjian_PIEEE_80}), we get:
%%%%%%%%%%%%%%%%%%%%%%%%%%%%%%%%%%%%%%%%%%%%%%%%%%
\be\ve{E}^{\rm m}(\ve{r}_i) = \ve{E}^0(\ve{r}_i) +\sum_{j=1,i\neq j}^N
\ve{T}(\ve{r}_i,\ve{r}_j) \chi(\ve{r}_j) d^3 \ve{E}^{\rm
  m}(\ve{r}_j)-\frac{4\pi}{3}\chi(\ve{r}_i) \ve{E}^{\rm
  m}(\ve{r}_i),\ee
%%%%%%%%%%%%%%%%%%%%%%%%%%%%%%%%%%%%%%%%%%%%%%%%%%
then we can write
%%%%%%%%%%%%%%%%%%%%%%%%%%%%%%%%%%%%%%%%%%%%%%%%%%
\be\ve{E}(\ve{r}_i) & = & \ve{E}^0(\ve{r}_i) +\sum_{j=1,i\neq j}^N
\ve{T}(\ve{r}_i,\ve{r}_j) \alpha_{\rm CM}(\ve{r}_j) \ve{E}(\ve{r}_j) \\
{\rm with} \phantom{000} \ve{E}(\ve{r}_i) & = &
\frac{\varepsilon(\ve{r}_i)+2}{3}
\ve{E}^{\rm m}(\ve{r}_i) \\
\alpha_{\rm CM}(\ve{r}_j) & = & \frac{3}{4\pi} d^3
\frac{\varepsilon(\ve{r}_i)-1}{\varepsilon(\ve{r}_i)+2} .\ee
%%%%%%%%%%%%%%%%%%%%%%%%%%%%%%%%%%%%%%%%%%%%%%%%%%
The field $\ve{E}(\ve{r}_i)$ is the local field, {\it i.e.}  the field
at the position $i$ in the absence of the subunit $i$.Then the linear
system can be written formally as
%%%%%%%%%%%%%%%%%%%%%%%%%%%%%%%%%%%%%%%%%%%%%%%%%%
\be \ve{E} = \ve{E}^0 + \ve{A} \ve{D}_\alpha \ve{E}, \label{eqmsym}\ee
%%%%%%%%%%%%%%%%%%%%%%%%%%%%%%%%%%%%%%%%%%%%%%%%%%
where $\ve{A}$ is a matrix which contains all the Green function and
$\ve{D}_\alpha$ is a tridiagonal matrix with the polarizabilities of
each element of discretization. In the next chapter we detail how to
solve Eq.~(\ref{eqmsym}) rigorously, but in this present chapter we
detail different approached methods to avoid the tedious resolution of
Eq.~(\ref{eqmsym}).  The scattered field is computed through
%%%%%%%%%%%%%%%%%%%%%%%%%%%%%%%%%%%%%%%%%%%%%%%%%%
\be\ve{E}^{\rm d}(\ve{r}) & = & \sum_{j=1}^N \ve{T}(\ve{r},\ve{r}_j)
\alpha(\ve{r}_j) \ve{E}(\ve{r}_j). \ee
%%%%%%%%%%%%%%%%%%%%%%%%%%%%%%%%%%%%%%%%%%%%%%%%%%



\section{Approximated method}


\subsection{Born}


The most simple approximation is the Born approximation which consists
to assume the field inside the object equal to the incident field for
each element of discretization:
%%%%%%%%%%%%%%%%%%%%%%%%%%%%%%%%%%%%%%%%%%%%%%%%%%
\be \ve{E}^{\rm m}(\ve{r}_i) = \ve{E}^0(\ve{r}_i), \ee
%%%%%%%%%%%%%%%%%%%%%%%%%%%%%%%%%%%%%%%%%%%%%%%%%%
This approximation hold if the contrast is weak and the object small
compare to the wavelength of illumination.


\subsection{Renormalized Born }

The renormalized Born approximation consists to assume the local field
inside the object equal to the incident field :
%%%%%%%%%%%%%%%%%%%%%%%%%%%%%%%%%%%%%%%%%%%%%%%%%%
\be \ve{E}(\ve{r}_i) = \ve{E}^0(\ve{r}_i). \ee
%%%%%%%%%%%%%%%%%%%%%%%%%%%%%%%%%%%%%%%%%%%%%%%%%%
In that case the macroscopic field reads:
%%%%%%%%%%%%%%%%%%%%%%%%%%%%%%%%%%%%%%%%%%%%%%%%%%
\be \ve{E}^{\rm m}(\ve{r}_i) = \frac{3}{\varepsilon(\ve{r}_i)+2}
\ve{E}^0(\ve{r}_i). \ee
%%%%%%%%%%%%%%%%%%%%%%%%%%%%%%%%%%%%%%%%%%%%%%%%%%
This approximation is better that the classical Born approximation
when the permittivity is high.


\subsection{Born at the order 1}

To be more precise that the renormalized Born approximation, one can
perform the Born series at the order one:
%%%%%%%%%%%%%%%%%%%%%%%%%%%%%%%%%%%%%%%%%%%%%%%%%%
\be\ve{E}(\ve{r}_i) & = & \ve{E}^0(\ve{r}_i) +\sum_{j=1,i\neq j}^N
\ve{T}(\ve{r}_i,\ve{r}_j) \alpha(\ve{r}_j) \ve{E}^0(\ve{r}_j). \ee
%%%%%%%%%%%%%%%%%%%%%%%%%%%%%%%%%%%%%%%%%%%%%%%%%%
In that case we take into account the simple scattering.


\subsection{Rytov}

The Rytov approximation consist to take into account the phase
variation inside the object:
%%%%%%%%%%%%%%%%%%%%%%%%%%%%%%%%%%%%%%%%%%%%%%%%%%
\be E_\beta^{\rm m}(\ve{r}_i) & = & E_\beta^0(\ve{r}_i) e^{E^{\rm
    d}_\beta(\ve{r}_i)/E^0_\beta(\ve{r}_i)}, \ee
%%%%%%%%%%%%%%%%%%%%%%%%%%%%%%%%%%%%%%%%%%%%%%%%%%
with $\beta=x,y,z$. Notice that when a component of the incident field
is null, then $E_\beta^{\rm m}=0$. This approximation permits to deal
with large object compare to the wavelength of illumination, but
always with low contrast of permittivity. The diffracted field reads:
%%%%%%%%%%%%%%%%%%%%%%%%%%%%%%%%%%%%%%%%%%%%%%%%%%
\be\ve{E}^{\rm d}(\ve{r}_i) & = & \sum_{j=1}^N
\ve{T}(\ve{r}_i,\ve{r}_j) \chi(\ve{r}_j) \ve{E}^0(\ve{r}_j), \ee
%%%%%%%%%%%%%%%%%%%%%%%%%%%%%%%%%%%%%%%%%%%%%%%%%%

\subsection{Renormalized Rytov }

The renormalized Rytov approximation deals with the local field:
%%%%%%%%%%%%%%%%%%%%%%%%%%%%%%%%%%%%%%%%%%%%%%%%%%
\be E_\beta(\ve{r}_i) & = & E_\beta^0(\ve{r}_i) e^{E_\beta^{\rm
    d}(\ve{r}_i)/E_\beta^0(\ve{r}_i)}, \ee
%%%%%%%%%%%%%%%%%%%%%%%%%%%%%%%%%%%%%%%%%%%%%%%%%%
and the diffracted field reads:
%%%%%%%%%%%%%%%%%%%%%%%%%%%%%%%%%%%%%%%%%%%%%%%%%%
\be\ve{E}^{\rm d}(\ve{r}_i) & = & \sum_{j=1,i\neq j}^N
\ve{T}(\ve{r}_i,\ve{r}_j) \alpha(\ve{r}_j) \ve{E}^0(\ve{r}_j). \ee
%%%%%%%%%%%%%%%%%%%%%%%%%%%%%%%%%%%%%%%%%%%%%%%%%%



\subsection{Beam propagation method (BPM)}

BPM is a class of algorithms designed for calculating the optical
field distribution in space for very large object compare to the
wavelength of illumination. BPM allows to obtain the electromagnetic
field via alternating evaluation of diffraction and refraction steps
handled in the Fourier and space domains It is important to note that
BPM ignores reflections, for more details see
Ref.~\onlinecite{Kamilov_IEEE_16}. In final the field reads
%%%%%%%%%%%%%%%%%%%%%%%%%%%%%%%%%%%%%%%%%%%%%%%%%
\be \ve{E}^{\rm m}(x,y,z+d)= e^{i k_0 n(x,y,z+d) d } {\cal
  F}^{-1}\left[ {\cal F} [\ve{E}^{\rm m}(x,y,z)] e^{-i(k_0-k_z) d}
\right], \ee
%%%%%%%%%%%%%%%%%%%%%%%%%%%%%%%%%%%%%%%%%%%%%%%%%
where the field at the position $(x,y,z+d)$ is computed with the
permittity at the same position and the field at the previous plane
$z$.  It is clear with this relation that the field is propagated only
in the direction of the positive $z$. Note that the size of the FFT is
given by the drop down menu and to avoid angle of incidence too
high. Notice that the diffracted field is computed like the other
methods which is more precise that the Kirchhoff's equation.

\subsection{Renormalized BPM}

We can do the same but with the local field:
%%%%%%%%%%%%%%%%%%%%%%%%%%%%%%%%%%%%%%%%%%%%%%%%%
\be \ve{E}(x,y,z+d)= e^{i k_0 n(x,y,z+d) d } {\cal F}^{-1}\left[ {\cal
    F} [\ve{E}(x,y,z)] e^{-i(k_0-k_z) d} \right]. \ee
%%%%%%%%%%%%%%%%%%%%%%%%%%%%%%%%%%%%%%%%%%%%%%%%%
