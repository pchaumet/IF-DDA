%%%%% Main pour appeler les sat %%%%%
\documentclass[english, a4paper, 11pt, oneside]{book}

%%%%%%%%%%%%%%%%%%%%%%%%%%%%%%%%%%%%%%%%%%%%%%%%%%%%%%
%%%%%% regle la mise en page, et les chapitres %%%%%%%
%%%%%%%%%%%%%%%%%%%%%%%%%%%%%%%%%%%%%%%%%%%%%%%%%%%%%%
\usepackage{fancyhdr}
\usepackage{babel}
\usepackage[english]{minitoc}
\usepackage{amssymb,amsbsy,amsfonts,amsmath,subeqnarray,eqnarray}
\usepackage{amsfonts}
\usepackage{vmargin}	% gere les marges
\usepackage{color}  % gere toute les couleurs
\usepackage{float}
\usepackage{picins}
\usepackage{graphicx}
\usepackage{program}
\usepackage{hyperref}
\usepackage{atxy}

\hypersetup{
     backref=true,    %permet d'ajouter des liens dans...
     pagebackref=true,%...les bibliographies
     hyperindex=true, %ajoute des liens dans les index.
     colorlinks=true, %colorise les liens
     breaklinks=true, %permet le retour a la ligne dans les liens trop longs
     urlcolor= blue,  %couleur des hyperliens
     linkcolor= blue, %couleur des liens internes
     bookmarks=true,  %cree des signets pour Acrobat
     bookmarksopen=true,            %si les signets Acrobat sont crees,
                                    %les afficher completement.
     pdftitle={Userguide IFDDA}, %informations apparaissant dans
     pdfauthor={Patrick C. Chaumet},     %dans les informations du document
     pdfsubject={IFDDA}          %sous Acrobat.
}



%%%%%%%%% declaration pour references comme ds revtex 4 %%%%%%%
\usepackage[numbers,super,sort&compress]{natbib}
\makeatletter \DeclareRobustCommand\onlinecite{\@onlinecite}
\def\@onlinecite#1{\begingroup\let\@cite
\NAT@citenum\citealp{#1}\endgroup} \makeatother
%%%%%%%%% change numerotation des footnotes %%%%%%%%%%%%%%%%%
\renewcommand{\thefootnote}{\roman{footnote}}

% A4wide pour elargir la page....

%%%% redefini les captions
\usepackage[small]{caption2}
\renewcommand{\captionfont}{\it \small}
\renewcommand{\captionlabelfont}{\it \bf \small}
\renewcommand{\captionlabeldelim}{ :}
\setlength{\captionmargin}{20 pt}%\captionmargin
%%%%%%%%%%%%%%%%%%%%%%%%%%%%%%%%%%%%%%%%%%%%%%%%%%%%%%
%%%%%%%%%%%%%%%%% regle des marges %%%%%%%%%%%%%%%%%%%
%%%%%%%%%%%%%%%%%%%%%%%%%%%%%%%%%%%%%%%%%%%%%%%%%%%%%%
\setmargins{25mm}{16mm}{150mm}{240mm}{10mm}{5mm}{10mm}{10mm}
%           left  top   width  height head  hsep foot  fskip
\setcounter{secnumdepth}{7}
\setcounter{tocdepth}{7}
\setcounter{minitocdepth}{2}
%\setcounter{lofdepth}{2}
%\setlength{\doublerulesep}{\arrayrulewidth} %% pour les tableaux(E)
%%%%%%%%%%%%%%%%%%%%%%%%%%%%%%%%%%%%%%%%%%%%%%%%%%%%%%%

%%%%%%%%%%%%%%%%%%%%%%%%%%%%%%%%%%%%%%%%%%%%%%%%%%%%%%
%%%%%%%%%%%%%%%%%% style de la page %%%%%%%%%%%%%%%%%%
%%%%%%%%%%%%%%%%%%%%%%%%%%%%%%%%%%%%%%%%%%%%%%%%%%%%%%
\definecolor{gris}{gray}{0.50}
\pagestyle{fancy}
\fancyhf{}
\renewcommand{\chaptermark}[1]{\markboth{#1}{}}
\renewcommand{\sectionmark}[1]{\markright{\thesection\ #1}}
\fancyhead[LE,RO]{\bfseries\thepage}
\fancyhead[LO,RE]{\bfseries\footnotesize\textcolor{gris}{\rightmark}}
%%%%%%%%%%%%%%%%%%%%%%%%%%%%%%%%%%%%%%%%%%%%%%%%%%%%

%%%%%%%%%%%%%%%%%%%%%%%%%%%%%%%%%%%%%%%%%%%%%%%%%%%%
%%%%%%%%%%%%%%% style des chapitres %%%%%%%%%%%%%%%%
%%%%%%%%%%%%%%%%%%%%%%%%%%%%%%%%%%%%%%%%%%%%%%%%%%%%
\makeatletter
\def\thickhrulefill{\leavevmode \leaders \hrule height 1ex \hfill \kern \z@}
\def\@makechapterhead#1{%
  %\vspace*{50\p@}%
  \vspace*{10\p@}%
  {\parindent \z@ \centering \reset@font
        \thickhrulefill\quad
        \scshape \@chapapp{} \thechapter
        \quad \thickhrulefill
        \par\nobreak
        \vspace*{10\p@}%
        \interlinepenalty\@M
        \hrule
        \vspace*{10\p@}%
        \Huge \bfseries #1\par\nobreak
        \par
        \vspace*{10\p@}%
        \hrule
    %\vskip 40\p@
    \vskip 100\p@
  }}
\def\@makeschapterhead#1{%
  %\vspace*{50\p@}%
  \vspace*{10\p@}%
  {\parindent \z@ \centering \reset@font
        \thickhrulefill
        \par\nobreak
        \vspace*{10\p@}%
        \interlinepenalty\@M
        \hrule
        \vspace*{10\p@}%
        \Huge \bfseries #1\par\nobreak
        \par
        \vspace*{10\p@}%
        \hrule
    %\vskip 40\p@
    \vskip 100\p@
  }}
%%%%%%%%%%%%%%%%%%%%%%%%%%%%%%%%%%%%%%%%%%%%%%%%%%%%
% L'environnement changemargin d�crit ci-dessous permet de
% modifier localement les marges d'un document. Il prend deux
% arguments, la marge gauche et la marge droite (ces arguments
% peuvent prendre des valeurs n�gatives).
%%%% debut macro %%%%
\newenvironment{changemargin}[2]{\begin{list}{}{%
\setlength{\topsep}{0pt}%
\setlength{\leftmargin}{0pt}%
\setlength{\rightmargin}{0pt}%
\setlength{\listparindent}{\parindent}%
\setlength{\itemindent}{\parindent}%
\setlength{\parsep}{0pt plus 1pt}%
\addtolength{\leftmargin}{#1}%
\addtolength{\rightmargin}{#2}%
}\item }{\end{list}}
%%%%%%%%%%%
\interfootnotelinepenalty=10000
%%%evite les orphelins
\widowpenalty=10000
\clubpenalty=10000
\raggedbottom



%%%% fin macro %%%%
%%%%%%%%%%%%%%%%%%%%%%%%%%%%%%%%%%%%%%%%%%%%%%%%%%%%
%%%%%%%%%%%%%% Debut du document %%%%%%%%%%%%%%%%%%%
%%%%%%%%%%%%%%%%%%%%%%%%%%%%%%%%%%%%%%%%%%%%%%%%%%%%
\begin{document}
\frontmatter 
\include{def}
\pagenumbering{roman}
%%%%%%%%%%%%%%%%%%%%%%%%%%%%%%%%%%%%%%%%%%%%%%%%%%%%%%
%%%%%%%%%%%%%%%%% PAGE DE GARDE  %%%%%%%%%%%%%%%%%%%%%
%%%%%%%%%%%%%%%%%%%%%%%%%%%%%%%%%%%%%%%%%%%%%%%%%%%%%%

%Une commande sembleble � \rlap ou \llap, mais centrant son argument
\def\clap#1{\hbox to 0pt{\hss #1\hss}}%
%Une commande centrant son contenu (� utiliser en mode vertical)
\def\ligne#1{%
  \hbox to \hsize{%
    \vbox{\centering #1}}}%
%Une comande qui met son premier argument � gauche, le second au 
%milieu et le dernier � droite, la premi�re ligne ce chacune de ces
%trois boites co�ncidant
\def\haut#1#2#3{%
  \hbox to \hsize{%
    \rlap{\vtop{\raggedright #1}}%
    \hss
    \clap{\vtop{\centering #2}}%
    \hss
    \llap{\vtop{\raggedleft #3}}}}%
%Idem, mais cette fois-ci, c'est la derni�re ligne
\def\bas#1#2#3{%
  \hbox to \hsize{%
    \rlap{\vbox{\raggedright #1}}%
    \hss
    \clap{\vbox{\centering #2}}%
    \hss
    \llap{\vbox{\raggedleft #3}}}}%
%La commande \maketitle
\def\maketitle{%
  \thispagestyle{empty}\vbox to \vsize{%
    \haut{}{\@blurb}{}
    \vspace{3cm}
   
    %\vfill
    \begin{center}\leavevmode
    	\normalfont
    	{\raggedleft \@author\par}%
    	%\thickhrulefill\par
    	\vspace{20mm} \hrule height 2pt 
    	{\huge\center \textbf{\@title}}%
    	\vspace{5mm} \hrule height 2pt \vspace{5mm}
	\vfill
    	\vskip 1cm
    	{\LARGE\center\textsc{}}
    	\vskip 2cm

    	
    \end{center}% 
    \vskip 1cm
    }%
  \cleardoublepage
  }

%Les commandes permettant de d�finir la date, le lieu, etc.
\def\date#1{\def\@date{#1}}
\def\author#1{\def\@author{#1}}
\def\title#1{\def\@title{#1}}
\def\location#1{\def\@location{#1}}
\def\blurb#1{\def\@blurb{#1}}
\def\email#1{\def\@email{#1}}
%Valeurs par d�faut
\date{\today}
\author{}
\title{}
\location{Marseille}
\blurb{}
\email{patrick.chaumet@fresnel.fr}
\makeatother
%
%%%%%%%%%%%%%%%%%%%%%%%%%%%%%%%%%%%%%%%%%%%%%%%%%%%%%%%%%%%%%%%%%%%%
\blurb{
\begin{center}
\parpic{
%\resizebox{160mm}{!}{\includegraphics{logofac.eps}}
}
\picskip{0}
\end{center}
 {\huge \textsc{Institut Fresnel}} 
}

\title{IF-DDA \\ \vspace{5mm} \textsc{Idiot Friendly-Discrete Dipole
    Approximation}\\ \vspace{5mm} {\Large version : 0.6.23}}

\author{\center{\LARGE \textsc{Patrick
      C. Chaumet} \\ \vspace{5mm} \textsc{Daniel Sentenac} \\
    \vspace{5mm} \textsc{Anne Sentenac}}}

\atxy(3cm,17.5cm){\resizebox{140mm}{!}{\includegraphics{schemamic.eps}}}



\email{patrick.chaumet@fresnel.fr}
\date{}


\newpage{\pagestyle{empty}\cleardoublepage}


\maketitle
\newpage{\pagestyle{empty}\cleardoublepage}
\newpage{\pagestyle{empty}\cleardoublepage}
\dominitoc 
\newpage{\pagestyle{empty}\cleardoublepage}
\tableofcontents
\clearpage{\pagestyle{empty}\cleardoublepage}
\addstarredchapter{List of figures}
\listoffigures
\clearpage{\pagestyle{empty}\cleardoublepage}
%\include{remerciements}
\mainmatter 
\pagenumbering{arabic}
\chapter{Generality}\label{chap1}
\markboth{\uppercase{Generality}}{\uppercase{Generality}}

\minitoc

\section{Introduction}


This software computes the diffraction of an electromagnetic wave by a
three-dimensional object. This interaction is taken into account
rigorously by solving the Maxwell's equations, but can also do with
the approximation of Born, Rytov or the BPM. The code has an
user-friendly interface and allows you to choose canonical objects
(sphere, cube, ...) as well as predefined incident waves (plane wave,
Gaussian beam, ...) or aribitrary objects and incidents waves. After
by drop-down menus, it is easy to study cross sections, optical forces
and torques, diffraction near field and far field as well as
microsocopy.


There are numerous methods that enable the study of the diffraction of
an electromagnetic wave by an object of arbitrary form and relative
permittivity. We are not going here to set up an exhaustive list of
these methods, but the curious reader may refer to the article by
F. M. Kahnert who details the advantages and weaknesses of the most
common methods.~\cite{Kahnert_JQSRT_03} 

The method we use is called coupled dipoles method (CDM) or the
discrete dipole approximation (DDA). This method is a volume method,
because the diffracted field is obtained from an integral, the support
of which is the volume of the considered object.  It had been
introduced by E. M. Purcell and C. R. Pennypacker in 1973, in order to
study the scattering of light by grains in interstellar
medium.~\cite{Purcell_AJ_73} The DDA has been subsequently widened to
objects in presence of a plane substrate or in a multilayer system,
see for instance Ref.~[\onlinecite{Rahmani_PRA_97}]. These past few
years, we endeavoured, on the one hand, to widen the DDA to more
complex geometries (grating with or without any default) and, on the
other hand, to increase its precision.  These improvements tend to
make this chapter a little technical, but they are going to be applied
in the next chapters. Before studying more in details the last
improvements of DDA, though, let us remind first of its principle.

\section{The principle of discrete dipole approximation}

Take an object of arbitrary form and relative permittivity in a
homogeneous space that we suppose here being the vacuum. This object
is submitted to an incident electromagnetic wave of wavelength
$\lambda$ ($k_0=2\pi/\lambda$).  The principle of the DDA consists in
representing the object as a set of $N$ small cubes of an edge $a$ [by
little, we mean smaller than the wavelength in the object :
$a\ll \lambda/\sqrt{\varepsilon}$ (Fig.~\ref{discretisation})].
%%%%%%%%%%%%%%%%%%%%%%%%%%%%%%%%%%%%%%%%%%%%%%%%%%%%%%%%%%%%%%%%%%%%%%
\begin{figure}
\begin{center}
\includegraphics*[draft=false,width=150mm]{discretisation.eps}
\caption{Principle of the DDA : the object under study (on the left)
  is discretized in a set of small dipoles (on the right)l.}
\label{discretisation}
\end{center}
\end{figure}
%%%%%%%%%%%%%%%%%%%%%%%%%%%%%%%%%%%%%%%%%%%%%%%%%%%%%%%%%%%%%%%%%%%%%%
Each one of the small cubes under the action of the incident wave is
going to get polarized, and as such, to acquire a dipolar moment,
whose value is going to depend on the incident field and on its
interaction with its neighbours. The local field of a dipole located
at $\ve{r}_i$, $\ve{E}(\ve{r}_i)$, is the sum of the incident wave and
the field radiated by the other $N-1$ dipoles :
%%%%%%%%%%%%%%%%%%%%%%%%%%%%%%%%%%%%%%%%%%%%%%%%%%%%%%%%%%%%%%%%%%%%%%
\be \label{cdms} \ve{E}(\ve{r}_i)=\ve{E}_0(\ve{r}_i)+\sum_{j=1,i\neq
j}^{N} \ve{T}(\ve{r}_i,\ve{r}_j)\alpha(\ve{r}_j)\ve{E}(\ve{r}_j). \ee
%%%%%%%%%%%%%%%%%%%%%%%%%%%%%%%%%%%%%%%%%%%%%%%%%%%%%%%%%%%%%%%%%%%%%%
$\ve{E}_0$ is the incident wave, $\ve{T}$ the linear susceptibility of
the field in homogeneous space:
%%%%%%%%%%%%%%%%%%%%%%%%%%%%%%%%%%%%%%%%%%%%%%%%%%%%%%%%%%%%%%%%%%%%%%
\be \ve{T}(\ve{r}_i,\ve{r}_j)=e^{ik_0 r}
\left[\left(3\frac{\ve{r}\bigotimes\ve{r}}{r^2}- \ve{I}\right)
  \left(\frac{1}{r^3}-\frac{ik_0}{r^2}\right) +
  \left(\ve{I}-\frac{\ve{r}\bigotimes\ve{r}}{r^2}\right)
  \frac{k_0^2}{r}\right] \ee
%%%%%%%%%%%%%%%%%%%%%%%%%%%%%%%%%%%%%%%%%%%%%%%%%%%%%%%%%%%%%%%%%%%%%%
with $\ve{I}$ the unity matrix and
$\ve{r}=\ve{r}_i-\ve{r}_j$. $\alpha$ is the polarizability of each
discretization element obtained from the Clausius-Mossotti
relation. Note that the polarizability $\alpha$, in order to respect
the optical theorem, needs to contain a term called the radiative
reaction term.~\cite{Draine_AJ_88} Equation~(\ref{cdms}) is valid for
$i=1,\cdots,N$, and so represents a system of $3N$ linear equations
where the local fields, $\ve{E}(\ve{r}_i)$, being the unknowns. Once
the system of linear equation is solved, the field scattered by the
object at an arbitrary position $\ve{r}$ is obtained by making the sum
of all the radiated fields by each one of the dipoles :
%%%%%%%%%%%%%%%%%%%%%%%%%%%%%%%%%%%%%%%%%%%%%%%%%%%%%%%%%%%%%%%%%%%%%%
\be \label{cdmd} \ve{E}(\ve{r})=\sum_{j=1}^{N} \ve{T}(\ve{r},\ve{r}_j)
\alpha(\ve{r}_j) \ve{E}(\ve{r}_j). \ee
%%%%%%%%%%%%%%%%%%%%%%%%%%%%%%%%%%%%%%%%%%%%%%%%%%%%%%%%%%%%%%%%%%%%%%
When the object is in presence of a plane substrate or within a
multilayer system, it is just necessary to replace $\ve{T}$, by the
linear susceptibility of the referential system.

We have just presented the DDA as E. M. Purcell and C. R. Pennypacker
had presented it earlier.~\cite{Purcell_AJ_73} Note that another
method very close to the DDA does exist. This method called the method
of the moments starts from the integral equation of Lippman Schwinger,
which is strictly identical to the DDA. The demonstration of the
equivalence between these two methods being a little technical, it is
explained in Ref.~\cite{Chaumet_PRE_04}.

The advantages of the DDA are that it is applicable to objects of
arbitrary forms, inhomogeneous (that is hardly achievable in case of
surface method), and anisotropic (the polarizability associated to the
mesh becomes a tensor). The outgoing wave condition is automatically
satisfied through the linear susceptibility of the field. Finally,
note that only the object is discretized unlike the methods of finite
differences and finite elements.~\cite{Kahnert_JQSRT_03} The main
inconvenience of the DDA consists in the fast increase of computation
time together with the increase of the number of discretization
elements, {\it i.e.}, the increase in size of the system of linear
equations to be solved.  There are ways to accelerate the resolution
of a system of linear equations very important in size as the method
of conjugated gradients, but, besides all, values of $N>10^6$ in
homogeneous space are difficult to deal with.

\section{A word about the code}

The code is thought to have a user-friendly interface so that everyone
can use it without any problems including non specialists. This allows
undergraduate students to study, for example, the basics of microscopy
(Rayleigh's criteria, notion of numerical aperture, ...)  or
diffraction without any problem; and researchers, typically
biologists, having no notion of Maxwell's equations to simulate what
gives a microscope (brightfield, phase microscope, dark field, ...) in
function of the usual parameters and the object. Nevertheless, this
code can also serve physicists specializing in electromagnetism in
performing, for example, calculations of optical forces, diffraction,
cross sections, near field and this with many incident beams.

The code thus has by default a simple interface where all numeroical
parameters are hidden and where many options are then chosen by
default. But it's easy to access all Code options by checking the
Advanced Interface option. This userguide explains how to use the
advanced interface in starting with the different approaches used by
the code to solve the Maxwell equations.

Note that the usability of the code is made to the detriment of the
optimization of the RAM and the code can used large memory for large
objects.


\section{How to compile the code}

The application is based on Qt and gfortran To install it you need :
qt, qt-devel, gcc-c++ et gfortran.  Notice that there is three
versions of the code, the first one is sequential and uses FFTE (Fast
Fourier Transform in the east), the second one uses FFTW (Fast Fourier
Transform in the west) which needs openmp 4.5 minimum and the third
uses HDF5 format to save data file. Currently according to the age of
the linux you use, you have Qt4 or Qt5. The code has been tested under
the two environments, but to compile you need adapt qt4 in qt5 on
recent versions, I will note for make compact qt4(5). Then to compile:

\begin{tabular}{|c|c|c|}
  \hline
  Code par défaut & Code avec FFTW & Code avec FFTW et HDF5 \\
  \hline
  qmake-qt4 & qmake-qt4 ``CONFIG+=fftw'' & qmake-qt4 ``CONFIG+=fftw hdf5'' \\
  make & make & make \\
make install & make install & make install \\
  \hline
\end{tabular}

To run the application, cd bin, and ./cdm.


On linux system with the library FFTW, it requires to install FFTW
packages with `` dnf install * fftw * ''. For the version that uses
HDF5 file you should install the following packages ``dnf install hdf
hdf5 hdf5-static hdf5-devel''.

The code works on windows system but it is tricky to compile it if you
want to use FFTW.

Notice that the code can be installed without the graphical interface.
Go to the directory tests, and then write ./comp (or /compfftw or
./compfftwhdf5 depending on the packages installed), then in the four
test directories four executables each are created corresponding to
four different configuration. It is quite clear that to change the
configuration you need to open the file main.f and change the options
inside the fortran, which is more tiedous than with the interface.
graphic.

\section{A word about the authors}

\begin{itemize}
\item P. C. Chaumet is Professor at Fresnel Institute of Aix-Marseille
  University, and deals with the development of the fortran source
  code.
\item A. Sentenac is research director at the CNRS, and works at
  Fresnel Institute of Aix-Marseille University, and participates to
  the development of the code connected to the far field diffraction.
\item D. Sentenac develops the convivial interface of the code.
\end{itemize}

\section{Licence}


Attribution-NonCommercial-ShareAlike 4.0 International (CC BY-NC-SA 4.0)

You are free to:

\begin{itemize}
\item Share - copy and redistribute the material in any medium or
  format
\item Adapt - remix, transform, and build upon the material
\end{itemize}

The licensor cannot revoke these freedoms as long as you follow the
license terms.
\begin{itemize}
\item Attribution - You must give appropriate credit, provide a link
  to the license, and indicate if changes were made. You may do so in
  any reasonable manner, but not in any way that suggests the licensor
  endorses you or your use.
\item NonCommercial - You may not use the material for commercial
  purposes.
\item ShareAlike - If you remix, transform, or build upon the
  material, you must distribute your contributions under the same
  license as the original.
\end{itemize}


\section{How to quote the code}

\begin{itemize}

\item If only the basic functions of the code are used:

P. C. {\textsc{Chaumet}}, A. {\textsc{Sentenac}}, and
A. {\textsc{Rahmani}}, \\{\it Coupled dipole method for scatterers
  with large permittivity.}\\
Phys. Rev. E {\bf 70}, 036606 (2004).

\item If the calculation of the optical forces is used, then:

P.C. {\textsc{Chaumet}}, A. {\textsc{Rahmani}},
A. {\textsc{Sentenac}}, and G. W. {\textsc{Bryant}},\\ {\it Efficient
  computation of optical forces with the coupled dipole method.}\\
Phys. Rev. E {\bf 72}, 046708 (2005).

\item If the calculation of optical couples is used:

P. C. {\textsc{Chaumet}} and C. {\textsc{Billaudeau}},\\ {\it Coupled
  dipole method to compute optical torque: Application to a
  micropropeller.}\\
J. Appl. Phys. {\bf 101}, 023106 (2007).

\item If the rigorous Gaussian beam is used:

P. C. {\textsc{Chaumet}},\\ {\it Fully vectorial highly non
  paraxial beam close to the waist.}\\
J. Opt. Soc. Am. A {\bf 23}, 3197 (2006).

\end{itemize}
   %    Generalites 
\chapter{Approximated method}\label{chapapprox}
\markboth{\uppercase{Approximated method}}{\uppercase{Approximated method}}

\minitoc

\section{Introduction}

In the previous chapter we have presented the DDA in a simple way
where the object under study is a set of radiating dipole. In an
approach more rigorous, with the Maxwell's equation, we get in
Gaussian unit:
%%%%%%%%%%%%%%%%%%%%%%%%%%%%%%%%%%%%%%%%%%%%%%%%%%
\be \venab \times \ve{E}^{\rm m}(\ve{r}) & = & i \frac{\omega}{c}
\ve{B}(\ve{r}) \\
\venab \times \ve{B}(\ve{r}) & = & -i \frac{\omega}{c}
\varepsilon(\ve{r}) \ve{E}^{\rm m}(\ve{r}), \ee
%%%%%%%%%%%%%%%%%%%%%%%%%%%%%%%%%%%%%%%%%%%%%%%%%%
where $\varepsilon(\ve{r})$ denotes the relative permittivity of the
object and $\ve{E}^{\rm m}$ the macroscopic field inside the object,
then we get
%%%%%%%%%%%%%%%%%%%%%%%%%%%%%%%%%%%%%%%%%%%%%%%%%%
\be \venab \times ( \venab \times \ve{E}^{\rm m}(\ve{r}) ) & = &
\varepsilon(\ve{r}) k_0^2 \ve{E}^{\rm m}(\ve{r}), \ee 
%%%%%%%%%%%%%%%%%%%%%%%%%%%%%%%%%%%%%%%%%%%%%%%%%%
with $k_0=\omega^2/c^2$. Using the relationship
$\varepsilon=1+4\pi \chi$, where $\chi$ denotes the linear field
susceptibility, we have:
%%%%%%%%%%%%%%%%%%%%%%%%%%%%%%%%%%%%%%%%%%%%%%%%%
\be \venab \times ( \venab \times \ve{E}^{\rm m}(\ve{r}) ) -k_0^2
\ve{E}^{\rm m}(\ve{r}) & = & 4\pi \chi(\ve{r}) k_0^2 \ve{E}^{\rm
  m}(\ve{r}) . \label{champref}\ee
%%%%%%%%%%%%%%%%%%%%%%%%%%%%%%%%%%%%%%%%%%%%%%%%%%
To solve this equation one needs the Green function defined as:
%%%%%%%%%%%%%%%%%%%%%%%%%%%%%%%%%%%%%%%%%%%%%%%%%%
\be \venab \times ( \venab \times \ve{T}(\ve{r},\ve{r}') ) -k_0^2
\ve{T}(\ve{r},\ve{r}') & = & 4\pi k_0^2 \ve{I}
\delta(\ve{r}-\ve{r}'), \ee
%%%%%%%%%%%%%%%%%%%%%%%%%%%%%%%%%%%%%%%%%%%%%%%%%%
and the solution of Eq.~(\ref{champref}) reads:
%%%%%%%%%%%%%%%%%%%%%%%%%%%%%%%%%%%%%%%%%%%%%%%%%%
\be\ve{E}^{\rm m}(\ve{r}) = \ve{E}_0(\ve{r}) +\int_{\Omega}
\ve{T}(\ve{r},\ve{r}') \chi(\ve{r}') \ve{E}^{\rm m}(\ve{r}') {\rm d}
\ve{r}',\ee
%%%%%%%%%%%%%%%%%%%%%%%%%%%%%%%%%%%%%%%%%%%%%%%%%%
where $\ve{E}^0$ is the incident field and $\Omega$ the support of the
object under study. When we solve Eq.~(\ref{champref}) the field
$\ve{E}^{\rm m}$ corresponds to macroscopic field inside the object.
To solve Eq.~(\ref{champref}) we discretize the object in a set of $N$
subunits with a cubic meshsize $d$, then the integral equation becomes
the sum of $N$ integrals:
%%%%%%%%%%%%%%%%%%%%%%%%%%%%%%%%%%%%%%%%%%%%%%%%%%
\be\ve{E}^{\rm m}(\ve{r}_i) = \ve{E}^0(\ve{r}_i) +\sum_{j=1}^{N}
\int_{V_j} \ve{T}(\ve{r}_i,\ve{r}') \chi(\ve{r}') \ve{E}^{\rm
  m}(\ve{r}') {\rm d} \ve{r}',\ee
%%%%%%%%%%%%%%%%%%%%%%%%%%%%%%%%%%%%%%%%%%%%%%%%%%
with $V_j=d^3$.  Assuming the field, the Green function and the
susceptibility constant over a subunit we get:
%%%%%%%%%%%%%%%%%%%%%%%%%%%%%%%%%%%%%%%%%%%%%%%%%%
\be\ve{E}^{\rm m}(\ve{r}_i) = \ve{E}^0(\ve{r}_i) +\sum_{j=1}^N
\ve{T}(\ve{r}_i,\ve{r}_j) \chi(\ve{r}_j) \ve{E}^{\rm m}(\ve{r}_j)
d^3.\ee
%%%%%%%%%%%%%%%%%%%%%%%%%%%%%%%%%%%%%%%%%%%%%%%%%%
Using, in first approximation (the radiative reaction term neglected)
$\int_{V_i}\ve{T}(\ve{r}_i,\ve{r}') {\rm d} \ve{r}'= -4\pi/3
$~\cite{Yaghjian_PIEEE_80}), we get:
%%%%%%%%%%%%%%%%%%%%%%%%%%%%%%%%%%%%%%%%%%%%%%%%%%
\be\ve{E}^{\rm m}(\ve{r}_i) = \ve{E}^0(\ve{r}_i) +\sum_{j=1,i\neq j}^N
\ve{T}(\ve{r}_i,\ve{r}_j) \chi(\ve{r}_j) d^3 \ve{E}^{\rm
  m}(\ve{r}_j)-\frac{4\pi}{3}\chi(\ve{r}_i) \ve{E}^{\rm
  m}(\ve{r}_i),\ee
%%%%%%%%%%%%%%%%%%%%%%%%%%%%%%%%%%%%%%%%%%%%%%%%%%
then we can write
%%%%%%%%%%%%%%%%%%%%%%%%%%%%%%%%%%%%%%%%%%%%%%%%%%
\be\ve{E}(\ve{r}_i) & = & \ve{E}^0(\ve{r}_i) +\sum_{j=1,i\neq j}^N
\ve{T}(\ve{r}_i,\ve{r}_j) \alpha_{\rm CM}(\ve{r}_j) \ve{E}(\ve{r}_j) \\
{\rm with} \phantom{000} \ve{E}(\ve{r}_i) & = &
\frac{\varepsilon(\ve{r}_i)+2}{3}
\ve{E}^{\rm m}(\ve{r}_i) \\
\alpha_{\rm CM}(\ve{r}_j) & = & \frac{3}{4\pi} d^3
\frac{\varepsilon(\ve{r}_i)-1}{\varepsilon(\ve{r}_i)+2} .\ee
%%%%%%%%%%%%%%%%%%%%%%%%%%%%%%%%%%%%%%%%%%%%%%%%%%
The field $\ve{E}(\ve{r}_i)$ is the local field, {\it i.e.}  the field
at the position $i$ in the absence of the subunit $i$.Then the linear
system can be written formally as
%%%%%%%%%%%%%%%%%%%%%%%%%%%%%%%%%%%%%%%%%%%%%%%%%%
\be \ve{E} = \ve{E}^0 + \ve{A} \ve{D}_\alpha \ve{E}, \label{eqmsym}\ee
%%%%%%%%%%%%%%%%%%%%%%%%%%%%%%%%%%%%%%%%%%%%%%%%%%
where $\ve{A}$ is a matrix which contains all the Green function and
$\ve{D}_\alpha$ is a tridiagonal matrix with the polarizabilities of
each element of discretization. In the next chapter we detail how to
solve Eq.~(\ref{eqmsym}) rigorously, but in this present chapter we
detail different approached methods to avoid the tedious resolution of
Eq.~(\ref{eqmsym}).  The scattered field is computed through
%%%%%%%%%%%%%%%%%%%%%%%%%%%%%%%%%%%%%%%%%%%%%%%%%%
\be\ve{E}^{\rm d}(\ve{r}) & = & \sum_{j=1}^N \ve{T}(\ve{r},\ve{r}_j)
\alpha(\ve{r}_j) \ve{E}(\ve{r}_j). \ee
%%%%%%%%%%%%%%%%%%%%%%%%%%%%%%%%%%%%%%%%%%%%%%%%%%



\section{Approximated method}


\subsection{Born}


The most simple approximation is the Born approximation which consists
to assume the field inside the object equal to the incident field for
each element of discretization:
%%%%%%%%%%%%%%%%%%%%%%%%%%%%%%%%%%%%%%%%%%%%%%%%%%
\be \ve{E}^{\rm m}(\ve{r}_i) = \ve{E}^0(\ve{r}_i), \ee
%%%%%%%%%%%%%%%%%%%%%%%%%%%%%%%%%%%%%%%%%%%%%%%%%%
This approximation hold if the contrast is weak and the object small
compare to the wavelength of illumination.


\subsection{Renormalized Born }

The renormalized Born approximation consists to assume the local field
inside the object equal to the incident field :
%%%%%%%%%%%%%%%%%%%%%%%%%%%%%%%%%%%%%%%%%%%%%%%%%%
\be \ve{E}(\ve{r}_i) = \ve{E}^0(\ve{r}_i). \ee
%%%%%%%%%%%%%%%%%%%%%%%%%%%%%%%%%%%%%%%%%%%%%%%%%%
In that case the macroscopic field reads:
%%%%%%%%%%%%%%%%%%%%%%%%%%%%%%%%%%%%%%%%%%%%%%%%%%
\be \ve{E}^{\rm m}(\ve{r}_i) = \frac{3}{\varepsilon(\ve{r}_i)+2}
\ve{E}^0(\ve{r}_i). \ee
%%%%%%%%%%%%%%%%%%%%%%%%%%%%%%%%%%%%%%%%%%%%%%%%%%
This approximation is better that the classical Born approximation
when the permittivity is high.


\subsection{Born at the order 1}

To be more precise that the renormalized Born approximation, one can
perform the Born series at the order one:
%%%%%%%%%%%%%%%%%%%%%%%%%%%%%%%%%%%%%%%%%%%%%%%%%%
\be\ve{E}(\ve{r}_i) & = & \ve{E}^0(\ve{r}_i) +\sum_{j=1,i\neq j}^N
\ve{T}(\ve{r}_i,\ve{r}_j) \alpha(\ve{r}_j) \ve{E}^0(\ve{r}_j). \ee
%%%%%%%%%%%%%%%%%%%%%%%%%%%%%%%%%%%%%%%%%%%%%%%%%%
In that case we take into account the simple scattering.


\subsection{Rytov}

The Rytov approximation consist to take into account the phase
variation inside the object:
%%%%%%%%%%%%%%%%%%%%%%%%%%%%%%%%%%%%%%%%%%%%%%%%%%
\be E_\beta^{\rm m}(\ve{r}_i) & = & E_\beta^0(\ve{r}_i) e^{E^{\rm
    d}_\beta(\ve{r}_i)/E^0_\beta(\ve{r}_i)}, \ee
%%%%%%%%%%%%%%%%%%%%%%%%%%%%%%%%%%%%%%%%%%%%%%%%%%
with $\beta=x,y,z$. Notice that when a component of the incident field
is null, then $E_\beta^{\rm m}=0$. This approximation permits to deal
with large object compare to the wavelength of illumination, but
always with low contrast of permittivity. The diffracted field reads:
%%%%%%%%%%%%%%%%%%%%%%%%%%%%%%%%%%%%%%%%%%%%%%%%%%
\be\ve{E}^{\rm d}(\ve{r}_i) & = & \sum_{j=1}^N
\ve{T}(\ve{r}_i,\ve{r}_j) \chi(\ve{r}_j) \ve{E}^0(\ve{r}_j), \ee
%%%%%%%%%%%%%%%%%%%%%%%%%%%%%%%%%%%%%%%%%%%%%%%%%%

\subsection{Renormalized Rytov }

The renormalized Rytov approximation deals with the local field:
%%%%%%%%%%%%%%%%%%%%%%%%%%%%%%%%%%%%%%%%%%%%%%%%%%
\be E_\beta(\ve{r}_i) & = & E_\beta^0(\ve{r}_i) e^{E_\beta^{\rm
    d}(\ve{r}_i)/E_\beta^0(\ve{r}_i)}, \ee
%%%%%%%%%%%%%%%%%%%%%%%%%%%%%%%%%%%%%%%%%%%%%%%%%%
and the diffracted field reads:
%%%%%%%%%%%%%%%%%%%%%%%%%%%%%%%%%%%%%%%%%%%%%%%%%%
\be\ve{E}^{\rm d}(\ve{r}_i) & = & \sum_{j=1,i\neq j}^N
\ve{T}(\ve{r}_i,\ve{r}_j) \alpha(\ve{r}_j) \ve{E}^0(\ve{r}_j). \ee
%%%%%%%%%%%%%%%%%%%%%%%%%%%%%%%%%%%%%%%%%%%%%%%%%%



\subsection{Beam propagation method (BPM)}

BPM is a class of algorithms designed for calculating the optical
field distribution in space for very large object compare to the
wavelength of illumination. BPM allows to obtain the electromagnetic
field via alternating evaluation of diffraction and refraction steps
handled in the Fourier and space domains It is important to note that
BPM ignores reflections, for more details see
Ref.~\onlinecite{Kamilov_IEEE_16}. In final the field reads
%%%%%%%%%%%%%%%%%%%%%%%%%%%%%%%%%%%%%%%%%%%%%%%%%
\be \ve{E}^{\rm m}(x,y,z+d)= e^{i k_0 n(x,y,z+d) d } {\cal
  F}^{-1}\left[ {\cal F} [\ve{E}^{\rm m}(x,y,z)] e^{-i(k_0-k_z) d}
\right], \ee
%%%%%%%%%%%%%%%%%%%%%%%%%%%%%%%%%%%%%%%%%%%%%%%%%
where the field at the position $(x,y,z+d)$ is computed with the
permittity at the same position and the field at the previous plane
$z$.  It is clear with this relation that the field is propagated only
in the direction of the positive $z$. Note that the size of the FFT is
given by the drop down menu and to avoid angle of incidence too
high. Notice that the diffracted field is computed like the other
methods which is more precise that the Kirchhoff's equation.

\subsection{Renormalized BPM}

We can do the same but with the local field:
%%%%%%%%%%%%%%%%%%%%%%%%%%%%%%%%%%%%%%%%%%%%%%%%%
\be \ve{E}(x,y,z+d)= e^{i k_0 n(x,y,z+d) d } {\cal F}^{-1}\left[ {\cal
    F} [\ve{E}(x,y,z)] e^{-i(k_0-k_z) d} \right]. \ee
%%%%%%%%%%%%%%%%%%%%%%%%%%%%%%%%%%%%%%%%%%%%%%%%%

\include{chappolaa}   %    Gestion des configurations
\include{chap2a}   %    Gestion des configurations
\include{chap3a}   %    Illumination
\include{chap4a}   %    D�finition de l'objet
\chapter{Possible study with the code}\label{chap5}
\markboth{\uppercase{Possible study with the
    code}}{\uppercase{Possible study with the code}}

\minitoc

\section{Introduction}

To determine the object with the appropriate orientation is not an
easy task.  That is why the first option {\it Only dipoles with
  epsilon}, enables us to check quickly if the object entered is well
the one intended without any calculation being launched. Once this has
been done, there are three great fields: the study in far field, the
study in near field and the optical forces.

\vskip10mm

{\underline{Important}}: Note that in the DDA the computation that
takes the longest time is the calculation of the local field due to
the necessity to solve the system of linear equations.  One option has
been added which consists in reading again the local field starting
with a file. When this option is selected, the name of a file is asked
for; either we enter an old file or a new name:

\begin{itemize}
\item If this is a new name, the calculation of the local field is
  going to be accomplished, then, stored together with the chosen
  configuration.
\item - If this is an old name, the local field is going to be read
  again with a checking that the configuration has not been changed
  between the writing and the second reading. This makes it easier to
  relaunch calculations very quickly for the same configuration but
  for different studies.
\end{itemize}


Note also that if the calculation asked has a large number of
discretization and that we are not interested by the output files in
.mat (needs to use matlab), then we have the option ``Do not write mat
file''. This requires the code to write no .mat file, and allows the
code to go faster, less fill the hard drive and be better
parallelized.


\section{Study in far field}

When the option far field is selected, three possibilities appear:

\begin{itemize}

\item {\it Cross section}: This option enables us to calculate the
  extinction ($C_{\rm ext}$), absorbing ($C_{\rm abs}$) and scattering
  cross section ($C_{\rm sca}$). The scattering cross section is
  obtained through $C_{\rm sca}=C_{\rm ext}-C_{\rm abs}$. The
  extinction and absorption cross sections may be evaluated as:
%%%%%%%%%%%%%%%%%%%%%%%%%%%%%%%%%%%%%%%%%%%%%%%%
  \be C_{\rm ext} & = & \frac{4\pi k_0}{\|\ve{E}_0\|^2} \sum_{j=1}^{N}
  {\rm Im} \left[ \ve{E}^*_0(\ve{r}_j).  \ve{p}(\ve{r}_j) \right] \\
  C_{\rm abs} & = & \frac{4\pi k_0}{\|\ve{E}_0\|^2} \sum_{j=1}^{N}
  \left[ {\rm Im} \left[ \ve{p}(\ve{r}_j). (\alpha^{-1}(\ve{r}_j))^*
      \ve{p}^*(\ve{r}_j) \right] -\frac{2}{3} k_0^3
    \| \ve{p}^*(\ve{r}_j) \|^2 \right] \ee
%%%%%%%%%%%%%%%%%%%%%%%%%%%%%%%%%%%%%%%%%%%%%%%%

\item {\it Cross section+Poynting}: This option calculates also the
  scattering cross section from the integration of the far field
  diffracted by the object upon 4$\pi$ steradians, the asymmetric
  factor and calculates differential cross section, {\it i.e.}
  $\left< \ve{S} \right> .\ve{n} R^2$ with $\ve{S}$ the Poynting
  vector, $\ve{n}$ the direction of observation, which is going to be
  represented in 3D. The values {\it Ntheta} and {\it Nphi} enable us
  to give the number of points used in order to calculate the
  scattering cross and to represent the Poynting vector. The larger
  the object is, the larger {\it Ntheta} and {\it Nphi} must be, which
  leads to time consuming calculations for objects of several
  wavelengths.
%%%%%%%%%%%%%%%%%%%%%%%%%%%%%%%%%%%%%%%%%%%%%%%%
  \be C_{\rm sca} & = & \frac{k_0^4}{\|\ve{E}_0\|^2} \int \left\|
    \sum_{j=1}^N \left[ \ve{p}(\ve{r}_j)-\ve{n}(\ve{n}.
      \ve{p}(\ve{r}_j)) \right] e^{-i k_0 \ve{n}.\ve{r}_j} \right\|^2
  {\rm d}\Omega \\ g & = & \frac{k_0^3}{C_{\rm sca} \|\ve{E}_0\|^2}
  \int \ve{n}.\ve{k}_0 \left\| \sum_{j=1}^N \left[
      \ve{p}(\ve{r}_j)-\ve{n}(\ve{n}.  \ve{p}(\ve{r}_j)) \right] e^{-i
      k_0 \ve{n}.\ve{r}_j} \right\|^2 {\rm d}\Omega \\
  \left< \ve{S} \right> .\ve{n} R^2 & = & \frac{c k_0^4}{8\pi }
  \left\| \sum_{j=1}^N \left[ \ve{p}(\ve{r}_j)-\ve{n}(\ve{n}.
      \ve{p}(\ve{r}_j)) \right] e^{-i k_0 \ve{n}.\ve{r}_j} \right\|^2
  \ee
%%%%%%%%%%%%%%%%%%%%%%%%%%%%%%%%%%%%%%%%%%%%%%%%
  

  Another solution in order to go faster (option {\it quick
    computation}) and to pass by FFT for the calculation of the
  diffracted field.  In this case, of course, it is convenient to
  discretize keeping in mind that the relation
  $\Delta x \Delta k=2\pi/N$ connects the mesh size of the
  discretization with the size of the FFT. The $N$ chosen for the
  moment is $N=256$. This is convenient for objects larger than the
  wavelength. Indeed, $L=N\Delta x$ corresponds to the size of the
  object which gives $\Delta k=2\pi/L$, and if the size of the object
  is too small, then, the $\Delta k$ is too large, and the quadrature
  is imprecise. Note that since the integration is performed on two
  planes parallel to the plane $(x,y)$, is not convenient if the
  incident makes an angle more than 70 degrees with the $z$ axis. The
  3D representation of the vector of Poynting is done as previously,
  i.e. with {\it Ntheta} and {\it Nphi} starting with an interpolation
  upon the calculated points with the FFT.

\item {\it Energy Conservation}. This study computes the reflectance,
  transmittance and absorptance. If the object under study is no
  absorbing then the absorptance should be zero. Then it traduces the
  level of energy conservation of our solver. It can depend of the
  precision of the iterative method and of the polarizability chosen.

\end{itemize}
  
\section{Microscopy}
  
This option permits to compute the image obtained for different
microscope (holographic, brightfield, darkfield and phase). It asks
for the numerical aperture of the objective lens (necessarily between
0 and 1), then, calculates the field diffracted by the object and the
picture obtained through the microscope. By default, the lenses are
placed parallel to the plane $(x,y)$ and at the side of the positive
$z$. The focus of the microscope is placed to the origin of the frame
but can be chaned via the field ``Position of the focal plane''.
(Fig.~\ref{lentille}). The magnification of the microscope is $G$ and
should be above 1.

\begin{figure}[h]
\begin{center}
\includegraphics*[draft=false,width=150mm]{lentille.eps}
\caption{Simplified figure of the microscope. The object focus of the
  objective lens is at the origin of the frame. The axis of the lens
  is confounded with the $z$ axis and at the side of the positive
  $z$.}
\label{lentille}
\end{center}
\end{figure}


The calculation for the diffracted field may be completed starting
with the sum of the radiation of the dipoles (very long when the
object has a lot of dipoles) or with FFT (option {\it quick
  computation}) with a value $N=128$ by default here as well. In this
case, $\Delta x \Delta k=2\pi/N$ with $\Delta x$ the mesh size of
discretization of the object which corresponds also to the
discretization of the picture plane. Consequently, this one has a size
of $L=N \Delta x$.

The diffracted field in far field at a distance $r$ of the origin can
written as
$\ve{E}= \ve{S}(k_x,k_y,\ve{r}_{\rm object}) \frac{e^{i k r}}{r}$. The
field after the first lens is then defined as:
$\ve{E}^f=\frac{\ve{S}(k_x,k_y,\ve{r}_{\rm object})}{-2 i \pi \gamma}$
with $\gamma=\sqrt{k_0^2-k_x^2-k_y^2}$ and the image through the
microscope is given by its Fourier transform,
$\ve{E}^i= {\cal F}(\ve{E}^f)$.

To take into account the magnification of the microscope in the image
we perform a rotation of the vector $\ve{E}^f$ before its Fourier
transform as:
%%%%%%%%%%%%%%%%%%%%%%%%%%%%%%%%%%%%%%%%%%
\be\ve{E}^i & = & {\cal F}(R(\theta) \ve{E}^f) \\
{\rm with~} R(\theta) & = & \left( \begin{matrix} u_x^2
    +\cos\theta(1-u_x^2) & u_x u_y (1-\cos\theta) & u_y \sin\theta \\
    u_x u_y (1-\cos\theta) & u_y^2 +\cos\theta(1-u_y^2) & -u_x
    \sin\theta \\ -u_y \sin\theta
    & u_x \sin\theta & \cos\theta  \end{matrix} \right) \\
\theta & = & \sin^{-1} [  \sin(-\beta)/G] - \beta \\
\beta & = & \cos^{-1}(k_z/k_0) \\
u_x & = & -k_y/k_{\parallel}\\
u_y & = & k_x/k_{\parallel} . \ee
%%%%%%%%%%%%%%%%%%%%%%%%%%%%%%%%%%%%%%%%
Note that we can simulate a microscope in tansmission $(k_z>0)$ or un
reflexion $(k_z<0)$.  We can notice, for a microscope in transmission,
that when the total field is computed in the Fourier plane (scattered
plus incident filed {\it i.e.} specular), in the case of the plane
wave, a Dirac in the Fourier space is placed at the pixel the closest
of the incident wave vector.


\begin{itemize}

\item {\it Holographic}: This option computes the diffracted field
  (Fourier plane) with the incident field defined in the section
  illumination properties. It computes the image plane with or without
  the presence of the incident field.


  
\item {\it Brightfield}: This microscope uses a condenser lens, which
  focuses light from the light source onto the sample with a numerical
  aperture defined below the magnification. It consists to sum many
  incident field inside this numerical aperture with different
  polarizations, hence it can take time as it needs to solve many
  direct problem. The result is given in the image plane with the
  incident field (a kind of dark field) and with the incident field
  (brightfield).

\item {\it Darkfield \& phase}: In darkfield microscopy the condenser
  is designed to form a hollow cone of light with a numerical aperture
  equal to the condenser lens, as apposed to brightfield microscopy
  that illuminates the sample with a full cone of light. The result is
  given in the image plane (scattered field). In the phase microscopy
  the ring-shaped illuminating light that passes the condenser annulus
  is focused on the specimen by the condenser exactly as in the dark
  field microscope and then the incident field with a phase shifted of
  $\pi/2$ is added to the scattered field.
  
\end{itemize}

\section{Study in near field}

When the option near field is selected, two possibilities appear:

\begin{itemize}

\item {\it Local field}: This option enables us to draw the local
  field to the position of each element of discretization. The local
  field being the field at the position of each element of
  discretization in absence of itself. 

\item {\it Macroscopic field}: This option enables us to draw the
  macroscopic field to the position of each element of
  discretization. The connection between the local field and the
  macroscopic field is given Ref.~\cite{Chaumet_PRE_04} :
%%%%%%%%%%%%%%%%%%%%%%%%%%%%%%
  \be \ve{E}_{\rm macro} & = & 3 \left( \varepsilon+2 -i \frac{k_0^3
      d^3 }{2 \pi} (\varepsilon-1)\right)^{-1} \ve{E}_{\rm local} \ee
%%%%%%%%%%%%%%%%%%%%%%%%%%%%%%


\end{itemize}

The last option enables us to choose the mesh in which the local and
macroscopic fields are represented.

\begin{itemize}

\item {\it Object}: Only the field in the object is
  represented. Notice that when FFT is used for the beam or for the
  computation of the diffracted field then this options is passed in
  the option {\it Cube}. This is same for the computation of the
  emissivity, teh reread option and the use of the BPM(R).

\item {\it Cube}: The field is represented within a cube containing
  the object.

\item {\it Wide field}: The field is represented within a box greater
  than the object.  The size of the box correspond to the size of the
  object plus the Additional sideband ($x$, $y$ ou $z$) on each
  side. For example for a sphere with a radius $r=100$~nm and
  discretization of 10, {\it i.e.} a meshsize of 10 nm, with an
  Additional sideband $x$ of 2, 3 for $y$ and 4 for $z$, we get a box
  of size:
%%%%%%%%%%%%%%%%%%%%%%%%%%%%%%%%%%%%%%%%%%
  \be l_x & = & 100 + 2\times 2 \times 10 = 140~{\rm nm} \\
  l_y & = & 100 + 2\times 3 \times 10 = 160~{\rm nm} \\
  l_z & = & 100 + 2\times 4 \times 10 = 180~{\rm nm} \\
  \ee
\end{itemize}

\section{Optical force and torque}

When the force option is selected, four possibilities appear:
\begin{itemize}

\item {\it Optical force}: Calculation of the optical force exerting
  on one or more objects.

\item {\it Optical force density}: Enables us to draw the density of
  the optical force.

\item {\it Optical torque}: Calculation of the optical torque exerting
  on one or more objects.  The torque is computed for an origin placed
  in the gravity center of the object.

\item {\it Optical torque density}: Enables us to draw the density of
  the optical force torque.
\end{itemize}
The net optical force and troque experienced by the object are
computed with~\cite{Chaumet_OL_00,Chaumet_JAP_07a}:
%%%%%%%%%%%%%%%%%%%%%%%%%%%%%%%%%%%%%%%%%%%%%%%%
\be \ve{F} & = & (1/2) \sum_{j=1}^N {\rm Re}\left(\sum_{v=1}^{3}
  p_v(\ve{r}_j) \frac{\partial (E_v(\ve{r}_j))^*}{\partial u}\right) \\
\ve{\Gamma} & = & \sum_{j=1}^N \left[ \ve{r}_{j} \times
  \ve{F}(\ve{r}^g_{j})+ \frac{1}{2} {\rm Re} \left\{ \ve{p}(\ve{r}_{j})
    \times \left[ \ve{p}(\ve{r}_{j})/{\alpha_{\rm
          CM}}(\ve{r}_{j})\right]^* \right\} \right].  \ee
%%%%%%%%%%%%%%%%%%%%%%%%%%%%%%%%%%%%%%%%%%%%%%%%
where $u$ or $v$, stand for either $x$ ,$y$, or $z$. The symbol $*$
denotes the complex conjugate. $\ve{r}^g_{j}$ is the vector bewteen
$j$ and the center of masse of the object.
 % Etude
\include{chap6a}   %    Repr�sentation des r�sultats 
\include{chap7a}   %    Output file for matlab
\chapter{Examples}\label{chaptest}
\markboth{\uppercase{Fichiers de test}}{\uppercase{Fichiers de test}}

\minitoc

\section{Introduction}

In bin/tests there is the file options.db3. You should copy it in the
directory bin as ''cp options.db3.. /.', and then you launch the code
after the load is appeara four test configurations that allow you to
see all the options in action.



\section{Test1}

The aim of test1 is to test a simple case and many options of the code
to validate them. Figure~\ref{test1conf} shows the options of the
chosen configuration.

%%%%%%%%%%%%%%%%%%%%%%%%%%%%%%%%%%%%%%%%%%%%%%%
\begin{figure}[H]
\begin{center}
  \includegraphics*[width=15.0cm,draft=false]{test1conf.eps}
\end{center}
\caption{Test1: configuration taken.}
\label{test1conf}
\end{figure}
%%%%%%%%%%%%%%%%%%%%%%%%%%%%%%%%%%%%%%%%%%%%%%%

The following figures show the results obtained. The plots are done
with Matlab and these are directly the eps files from the ifdda.m
script that are used. The advantage of matlab in this case is to give
all the figures in one go.

%%%%%%%%%%%%%%%%%%%%%%%%%%%%%%%%%%%%%%%%%%%%%%%
\begin{figure}[H]
\begin{center}
  \includegraphics*[width=15.0cm,draft=false]{test1local.eps}
\end{center}
\caption{Modulus of the local field in $(x,y)$ plane.}
%\label{test1res1}
\end{figure}
%%%%%%%%%%%%%%%%%%%%%%%%%%%%%%%%%%%%%%%%%%%%%%%
%%%%%%%%%%%%%%%%%%%%%%%%%%%%%%%%%%%%%%%%%%%%%%%
\begin{figure}[H]
\begin{center}
  \includegraphics*[width=15.0cm,draft=false]{test1macro.eps}
\end{center}
\caption{Modulus of the macroscopic field in $(x,y)$ plane.}
%\label{test1res1}
\end{figure}
%%%%%%%%%%%%%%%%%%%%%%%%%%%%%%%%%%%%%%%%%%%%%%%
Because the incident field is polarized along the $y$ direction (TE),
hence the $y$ component of the field inside the sphere is the largest.


%%%%%%%%%%%%%%%%%%%%%%%%%%%%%%%%%%%%%%%%%%%%%%%
\begin{figure}[H]
\begin{center}
  \includegraphics*[width=15.0cm,draft=false]{test1poynting2d.eps}
\end{center}
\caption{Modulus of the Poyting vector.}
%\label{test1res1}
\end{figure}
%%%%%%%%%%%%%%%%%%%%%%%%%%%%%%%%%%%%%%%%%%%%%%%
%%%%%%%%%%%%%%%%%%%%%%%%%%%%%%%%%%%%%%%%%%%%%%%
\begin{figure}[H]
\begin{center}
\begin{tabular}{cc}
  \includegraphics*[width=7.0cm,draft=false]{test1force2d.eps}
&  \includegraphics*[width=9.0cm,draft=false]{test1force3d.eps}
\end{tabular}

\end{center}
\caption{Optical force in the $(x,y)$ plane and in 3D.}
%\label{test1res2}
\end{figure}
%%%%%%%%%%%%%%%%%%%%%%%%%%%%%%%%%%%%%%%%%%%%%%%
%%%%%%%%%%%%%%%%%%%%%%%%%%%%%%%%%%%%%%%%%%%%%%%
\begin{figure}[H]
\begin{center}
\begin{tabular}{cc}
  \includegraphics*[width=7.0cm,draft=false]{test1torque2d.eps}
&  \includegraphics*[width=9.0cm,draft=false]{test1torque3d.eps}
\end{tabular}

\end{center}
\caption{Optical torque in the $(x,y)$ plane and in 3D.}
%\label{test1res2}
\end{figure}
%%%%%%%%%%%%%%%%%%%%%%%%%%%%%%%%%%%%%%%%%%%%%%%

%%%%%%%%%%%%%%%%%%%%%%%%%%%%%%%%%%%%%%%%%%%%%%%
\begin{figure}[H]
\begin{center}
\begin{tabular}{ccc}
  \includegraphics*[width=5.0cm,draft=false]{test1fourier.eps}
& \includegraphics*[width=5.0cm,draft=false]{test1image.eps}
&  \includegraphics*[width=5.0cm,draft=false]{test1imageinc.eps}
\end{tabular}

\end{center}
\caption{Microscopy in tramsmission: Modulus of the diffracted field
  in the Fourier plane (left), modulus of teh diffracted field in the
  image plane (middle), and modulus of the total field in the image
  plane (right).}
\end{figure}
%%%%%%%%%%%%%%%%%%%%%%%%%%%%%%%%%%%%%%%%%%%%%%%

\section{Test2}

The aim of the test2 is to test a simple case and many options code to
validate them.  Figure~\ref{test2conf}. shows the options of the
chosen configuration. The illumination is done by two plane waves.


%%%%%%%%%%%%%%%%%%%%%%%%%%%%%%%%%%%%%%%%%%%%%%%
\begin{figure}[H]
\begin{center}
  \includegraphics*[width=15.0cm,draft=false]{test2conf.eps}
\end{center}
\caption{Test2: configuration taken.}
\label{test2conf}
\end{figure}
%%%%%%%%%%%%%%%%%%%%%%%%%%%%%%%%%%%%%%%%%%%%%%%


%%%%%%%%%%%%%%%%%%%%%%%%%%%%%%%%%%%%%%%%%%%%%%%
\begin{figure}[H]
\begin{center}
\begin{tabular}{ccc}
  \includegraphics*[width=7.0cm,draft=false]{test2dipolepos.eps}
& \includegraphics*[width=7.0cm,draft=false]{test2epsilon.eps}
\end{tabular}

\end{center}
\caption{Object in 3D (left) and map of permittivity in the $(x,y)$
  plane (right).}
\end{figure}
%%%%%%%%%%%%%%%%%%%%%%%%%%%%%%%%%%%%%%%%%%%%%%%


The following figures show the results obtained.
%%%%%%%%%%%%%%%%%%%%%%%%%%%%%%%%%%%%%%%%%%%%%%%
\begin{figure}[H]
\begin{center}
  \includegraphics*[width=15.0cm,draft=false]{test2incident.eps}
\end{center}
\caption{Modulus of the incident field in $(x,y)$ plane.}
%\label{test1res1}
\end{figure}
%%%%%%%%%%%%%%%%%%%%%%%%%%%%%%%%%%%%%%%%%%%%%%%
%%%%%%%%%%%%%%%%%%%%%%%%%%%%%%%%%%%%%%%%%%%%%%%
\begin{figure}[H]
\begin{center}
  \includegraphics*[width=15.0cm,draft=false]{test2local.eps}
\end{center}
\caption{Modulus of the local field in $(x,y)$ plane.}
%\label{test1res1}
\end{figure}
%%%%%%%%%%%%%%%%%%%%%%%%%%%%%%%%%%%%%%%%%%%%%%%
%%%%%%%%%%%%%%%%%%%%%%%%%%%%%%%%%%%%%%%%%%%%%%%
\begin{figure}[H]
\begin{center}
  \includegraphics*[width=15.0cm,draft=false]{test2macro.eps}
\end{center}
\caption{Modulus of the macroscopic field in $(x,y)$ plane.}
%\label{test1res1}
\end{figure}
%%%%%%%%%%%%%%%%%%%%%%%%%%%%%%%%%%%%%%%%%%%%%%%

%%%%%%%%%%%%%%%%%%%%%%%%%%%%%%%%%%%%%%%%%%%%%%%
\begin{figure}[H]
\begin{center}
  \includegraphics*[width=15.0cm,draft=false]{test2poynting2d.eps}
\end{center}
\caption{Modulus of the Poynting vector.}
%\label{test1res1}
\end{figure}
%%%%%%%%%%%%%%%%%%%%%%%%%%%%%%%%%%%%%%%%%%%%%%%


%%%%%%%%%%%%%%%%%%%%%%%%%%%%%%%%%%%%%%%%%%%%%%%
\begin{figure}[H]
\begin{center}
\begin{tabular}{ccc}
  \includegraphics*[width=7.0cm,draft=false]{test2fourier.eps}
& \includegraphics*[width=7.0cm,draft=false]{test2image.eps}
\end{tabular}

\end{center}
\caption{Microscopy in reflexion: Modulus of the diffracted field in
  the fourier plane (left) and in image plane (right).}
\end{figure}
%%%%%%%%%%%%%%%%%%%%%%%%%%%%%%%%%%%%%%%%%%%%%%%




\section{Test3}

The aim of the test3 is to test the microscopy in dark field and
bright field in transmission. One studies a sphere with a radius of
500~nm and permittivity 1.5. 



%%%%%%%%%%%%%%%%%%%%%%%%%%%%%%%%%%%%%%%%%%%%%%%
\begin{figure}[H]
\begin{center}
  \includegraphics*[width=15.0cm,draft=false]{test3conf.eps}
\end{center}
\caption{Test3: configuration taken.}
\label{test3conf}
\end{figure}
%%%%%%%%%%%%%%%%%%%%%%%%%%%%%%%%%%%%%%%%%%%%%%%


%%%%%%%%%%%%%%%%%%%%%%%%%%%%%%%%%%%%%%%%%%%%%%%
\begin{figure}[H]
\begin{center}
\begin{tabular}{ccc}
 \includegraphics*[width=5.0cm,draft=false]{test3angleincbf.eps}
&  \includegraphics*[width=5.0cm,draft=false]{test3imagewf.eps}
& \includegraphics*[width=5.0cm,draft=false]{test3imageincwf.eps}
\end{tabular}

\end{center}
\caption{Microscopy in transmission: incident taken to make the image
  (left). Modulus of the diffracted field in the image plane (middle),
  and modulus of the total field inthe image plane (right).}
\end{figure}
%%%%%%%%%%%%%%%%%%%%%%%%%%%%%%%%%%%%%%%%%%%%%%%


\section{Test4}

Same configuration as in test3 for a dark field microscope.


%%%%%%%%%%%%%%%%%%%%%%%%%%%%%%%%%%%%%%%%%%%%%%%
\begin{figure}[H]
\begin{center}
  \includegraphics*[width=15.0cm,draft=false]{test4conf.eps}
\end{center}
\caption{Test4: configuration taken.}
\label{test4conf}
\end{figure}
%%%%%%%%%%%%%%%%%%%%%%%%%%%%%%%%%%%%%%%%%%%%%%%


%%%%%%%%%%%%%%%%%%%%%%%%%%%%%%%%%%%%%%%%%%%%%%%
\begin{figure}[H]
\begin{center}
\begin{tabular}{cc}
 \includegraphics*[width=7.0cm,draft=false]{test4angleincdf.eps}
&  \includegraphics*[width=7.0cm,draft=false]{test4imagewf.eps}
\end{tabular}

\end{center}
\caption{Microscope in reflexion: incident taken to make the image
  (left). Modulus of the diffracted field in the image plane (right).}
\end{figure}
%%%%%%%%%%%%%%%%%%%%%%%%%%%%%%%%%%%%%%%%%%%%%%%
   %    test

\addstarredchapter{Bibliography}
\markboth{\uppercase{Bibliography}}{\uppercase{Bibliography}}

\bibliographystyle{apsrev} 
\bibliography{bibliographie}



\end{document}
